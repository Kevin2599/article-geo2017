% \documentclass[manuscript]{geophysics}
% \documentclass[paper,twocolumn,twoside]{geophysics}
\documentclass[paper]{geophysics}
% \documentclass[manuscript,revised]{geophysics}
%%fakesection ===    PACKAGES & DEFINITIONS    ===

% \usepackage{draftwatermark}
% \SetWatermarkLightness{.95}
% \SetWatermarkFontSize{2cm}
% % \SetWatermarkText{\shortstack[c]{Submitted to \texttt{Geophysics}\\23 Oct 2012}}
% \SetWatermarkText{\shortstack[c]{Re-submitted to \texttt{Geophysics}\\05 Feb 2013}}

%% ~ ADDITIONAL PACKAGES TO GEOPHYSICS.CLS
\DeclareGraphicsExtensions{.pdf,.png,.jpg}
\usepackage{color}
\usepackage{tabularx}
\usepackage{colortbl}
\usepackage{booktabs}
\usepackage[USenglish]{babel}
\usepackage[utf8]{inputenc}
\usepackage{lmodern}
\usepackage[T1]{fontenc}
\usepackage{amssymb, amsmath, amsfonts}
% \usepackage[pdftex, draft]{hyperref}
\usepackage{xspace}                 % For spaces after \newcommand-strings
\usepackage{upquote}

% vvv STUFF NOT FOR SUBMISSION
\usepackage[pdftex, final]{hyperref}
\definecolor{myblue}{rgb}{0,0,.5}
\hypersetup{allcolors=myblue, allbordercolors={0 0 .5},
            pdfborderstyle={/S/U/W .5}, colorlinks=true,}
% ^^^ STUFF NOT FOR SUBMISSION

% Figure directory
\renewcommand{\figdir}{figures}
\usepackage[strings]{underscore}

%
% Figure and table widths
%   SEGTex               column width = 250 pt
%   Geophysics           column width = 240 pt (3.33 in)
%   Geophysics       1.3 column width = 312 pt ( 4.33 in)
% \plot[btp]{figure1}{width=240pt}{Diff porosities higher than roughly 40~\%.}
% \plot*[btp]{figure11}{width=\textwidth}{(a) i horizontal resistivities.}
%
\definecolor{MyGray}{gray}{0.85}
\newcolumntype{Y}{>{\columncolor{MyGray}\raggedleft\arraybackslash}r}
\newcolumntype{W}{>{\raggedleft\arraybackslash}r}
\newcolumntype{A}{>{\columncolor{MyGray}\raggedleft\arraybackslash}X}
\newcolumntype{B}{>{\raggedleft\arraybackslash}X}

\hyphenation{iso-tro-pic pe-ne-tra-ting}

\begin{document}

\title{An open-source \old{1D}\new{full 3D} electromagnetic modeler \new{for 1D VTI media} in Python: empymod}

\renewcommand{\thefootnote}{1}% \fnsymbol{footnote}}

\ms{GEO-2016-0626.R2}

\address{Instituto Mexicano del Petróleo,
         Eje Central Lázaro Cárdenas Norte 152,
         Col. San Bartolo Atepehuacan C.P. 07730,
         Ciudad de México, México.
         E-mail: \href{mailto:dieter@werthmuller.org}{Dieter@Werthmuller.org}.}

\author{Dieter Werthmüller\footnotemark[1]}

\footer{}
\lefthead{Werthmüller}
\righthead{Open-source\old{ 1D} EM modeler in Python}

\maketitle

%%fakesection ===    ABSTRACT    ===
\begin{abstract} % 1-2 sentence(s) each
  The presented Python-code empymod computes the three-dimensional
  electromagnetic field in a layered-earth with vertical transverse isotropy by
  combining and extending two earlier presented algorithms in this journal.
  The bottleneck in frequency- and time-domain calculations of electromagnetic
  responses derived in the wavenumber-frequency domain is the transformations
  from wavenumber to space domain and from frequency to time domain, the
  so-called Hankel and Fourier transforms. Three different Hankel transform
  methods (quadrature, quadrature-with-extrapolation, and filters) and four
  different Fourier transform methods (FFT, FFTLog,
  quadrature-with-extrapolation, and filters) are included in empymod, which
  allows to compare these different methods in terms of speed and precision.
  The best transform in terms of speed and precision depends on the modeled
  frequencies: \old{available filters}\new{published digital filters for the
  Hankel transform}, for instance, are very fast and precise for frequencies in
  the range of controlled-source electromagnetic data, but fail in the
  frequency range of ground penetrating radar. Conventional quadrature, on the
  other hand, is in comparison very slow but can model any frequency.  Examples
  comparing empymod with analytical solutions and with existing electromagnetic
  modelers illustrate the capabilities of empymod.
\end{abstract}

\section{Introduction}

% 1.1 CSEM in hydrocarbon exploration
The potential of electromagnetic methods for the detection of hydrocarbon
reservoirs is known for some decades, see for instance \cite{PIEEE.89.Nekut} or
\cite{B.SEG.91.Chave}. More recent, good overviews of the methodology and its
applications are give by \cite{SG.05.Edwards} and \cite{IEEE.12.Ziolkowksi}.
Whereas in the early days everything happened in idealized, isotropic one
dimensional (1D) earth models, it is generally agreed that the often complex
geology of hydrocarbon reservoirs requires two dimensional (2D) and three
dimensional (3D), anisotropic forward modelers. However, 1D models are still
very important. Besides that their calculation is very fast they allow to study
single, isolated effects to the electromagnetic field, which is a crucial
foundation in understanding the electromagnetic field behaviour and a necessity
for understanding the phenomena at higher dimensions. 1D models are furthermore
often used in inversion routines of higher dimensions, for instance to generate
a starting model or to calculate the primary fields for 2D/3D modeling and
inversion routines.

% 1.2 Existing 1D solutions
Solutions for the electromagnetic fields in a layered-earth model have been
solved and published extensively using different approaches. The importance and
wide\-spread use of 1D models is shown in the continuous stream of publications
in this area, even in recent years. \cite{GJI.07.Loseth} solved the problem
with the scattering matrix formulation for a 1D earth with general anisotropy,
spanning the frequency range from controlled-source electromagnetics (CSEM) to
ground-pe\-ne\-tra\-ting radar (GPR). \cite{GJI.09.Chave} presented a solution
in terms of independent and unique transverse electric (TE) and tangential
magnetic (TM) modes for the electrical isotropic case using the diffusive
approximation (without displacement currents, valid for low frequencies such as
in CSEM). \cite{GEO.09.Key} demonstrated why 1D models still matter by testing,
for instance, the benefit of additional frequencies in an inversion routine; he
follows and extends the magnetic vector potential approach by
\cite{B.AP.82.Wait} for forward modeling, using the isotropic, diffusive
low-frequency approach. \cite{GEO.15.Hunziker} obtained the electromagnetic
field in a layered earth with vertical transverse iso\-tro\-py (VTI) by solving
two equivalent scalar equations with a scalar global reflection coefficient.
All of the four citations regarding 1D solutions have quite extensive reference
lists about the history of 1D forward modeling, the last one featuring an
interesting review of the history of 1D electromagnetic derivations spanning
almost 200 years.

% 1.3 Hankel transform
A crucial as well as very interesting part of electromagnetic modeling is, once
one moves from the theoretical derivation to the numerical implementation, the
Hankel transform involved in the transformation from the
wave\-num\-ber-fre\-que\-ncy domain to the space-frequency domain. Because
what is common in all these approaches is that they derive solutions in the
wavenumber-frequency domain. This in turn requires a Hankel transform, which is
a numerically expensive, infinite integral containing oscillating, slowly
decaying Bessel functions (the Hankel transform is also known as the
Fourier-Bessel transform). The use of digital filters is quite common in
geophysics, known as the fast Hankel transform method (FHT), as introduced by
\cite{GP.71.Gosh}, and popularized by \cite{TRP.75.Anderson, GEO.79.Anderson,
TMS.82.Anderson} thanks to his freely available Fortran routines. Naturally,
standard quadrature can be used as well for the Hankel transform, see for
instance \cite{GEO.83.Chave}, and hybrid routines using both methods were
published by \cite{GEO.84.Anderson, GEO.89.Anderson}. The topic got picked up
again recently with new filters being published by \cite{GP.07.Kong} and
\cite{GEO.09.Key, GEO.12.Key}

% 1.4 1D codes and licenses
In terms of freely available and open-source code there are a few examples.
\cite{GEO.09.Key} published his forward modeling and inversion code Dipole1D,
written in Fortran. The dipole can be placed anywhere in the stack of layers
and can have arbitrary orientation and dip. Dipole1D computes the electric and
magnetic fields to an electric source, using the FHT method. \cite{GEO.12.Key}
published additional code with his introduction of the
qua\-dra\-ture-with-ex\-tra\-po\-la\-tion method (QWE) and its comparison to
the FHT method, for which he translated some of the Dipole1D-Fortran code to
Matlab. \cite{GEO.15.Hunziker} published their code EMmod (Fortran and C), in
which the source and receiver can also be placed anywhere in the stack of
layers. They use a 61\,pt Gauss-Kronrod quadrature for the Hankel transform.

% 1.5 Claim
With empymod I present a 1D forward modeling code that is based on
\cite{GEO.15.Hunziker} for the wavenumber-frequency domain calculation, and
mainly on \cite{GEO.12.Key} for the Hankel and Fourier transforms. In addition
to QWE and FHT, empymod includes an adaptive quadrature (QUAD) for the Hankel
transform, and the standard fast Fourier transform FFT as well as the
logarithmic fast Fourier transform FFTLog of \cite{RAS.00.Hamilton} for the
Fourier transform.

To my knowledge, empymod is the first code that is freely available which
calculates the full wavefield for a layered-earth model with vertical isotropy
and has different types of Hankel transforms (QWE, QUAD, FHT) and various
methods for the Fourier transform (FFTLog, sine/cosine-filters, QWE, FFT). An
interesting use-case for empymod is therefore comparison studies of the
different methods, up to building hybrid inversion schemes. The code is
published under the lax permissive \emph{Apache Version 2.0 license}, which
makes it available to everyone for free, even for commercial purposes. It
\old{therefore} might \new{hence} prove to be valuable for students and
professionals alike, \new{and it is valuable for the objectives of reproducible
research.} It is written in Python, a modern, cross-platform, free and
open-source programming language. With its scientific libraries, mainly the
numeric and scientific modules \texttt{NumPy} and \texttt{SciPy}, it creates an
extremely powerful numerical calculation stack: Even though it is an
interpreted language it can be very fast, as \texttt{NumPy}/\texttt{SciPy} use
under the hood routines in Fortran or in C (using for instance the BLAS/Lapack
libraries) for the linear algebra computations, and Python is merely the glue.
The comparisons show that empymod is as fast as the existing codes. The code is
vectorized (avoiding loops) as much as possible, which improves computation.
The most time-consuming calculations have furthermore a flag to run
parallelized.  However, on a larger scale such as an inversion routine one
would probably want to parallelize the kernel calls instead of the kernel
itself. The code is hosted on GitHub, which makes it easy for anyone to improve
and contribute to the code\new{.}\old{: https://github.com/prisae/empymod}
\new{It comes with an extensive texting suite which is run automatically on
each commit, which guarantees some robustness and helps to avoid regression
bugs. Installation guidelines, documentation, and examples of its usage can be
found on its website at} \url{https://empymod.github.io}.%
\old{The documentation is hosted on https://empymod.readthedocs.io}
\cite{TLE.17.Werthmuller} gives a tutorial-style introduction to
\old{EM}\new{CSEM} modeling using empymod.

After introducing the code in the first part I will present some comparisons by
reproducing results from the publications this code is based upon: First some
comparisons to the analytical half-space solution, followed by a comparison to
EMmod by \cite{GEO.15.Hunziker}, and finally a comparison to some results
presented by \cite{GEO.12.Key}. Last is a time-domain example in which I
compare the four different Fourier transforms.


\section{About the code}

The code consists of 5 files; these are 3 core modules plus \emph{utils}, which
contains input checks and other utilities, and \emph{filters}, containing the
FHT filter coefficients. The three core routines are: (1) \emph{kernel}, where
the wavenumber-domain calculation is carried out; (2) \emph{transform}, where
the Hankel and Fourier transforms are computed; and (3) \emph{model}, which
contains the actual modeling routines for end users. The main modeling routine
is \emph{bipole}, which can calculate frequency- and time-domain responses for
arbitrary oriented, electric or magnetic bipole sources and receivers of finite
length \new{(see also Appendix~\ref{apd:code})}.

The wavenumber-domain calculation in \emph{kernel} follows
\cite{GEO.15.Hunziker}, and calculates as such the complete wavefield for a
layered VTI model. The code makes no assumptions about the model: Source and
receiver can be placed anywhere in the model, including first and last layer,
and you have to define if the first layer is air or not. Bipoles can cross
layer boundaries. Depths, frequencies, and source-receiver configuration have
to be defined, and each layer is characterized with its horizontal resistivity
$\rho_\textrm{h}$, its electrical anisotropy $\lambda$, where $\lambda =
\sqrt{\rho_\textrm{v}/\rho_\textrm{h}}$, its horizontal and vertical magnetic
permeabilities $\mu_\mathrm{h}$ and $\mu_\mathrm{v}$, and the horizontal and
vertical electric permittivities $\epsilon_\mathrm{h}$ and
$\epsilon_\mathrm{v}$. The main difference between EMmod and empymod is the
Hankel transform, and a few additional fundamental differences: EMmod uses 2nd
order Bessel functions of the first kind $J_2$. Published FHT filters
generally provide coefficients for 0th and 1st order Bessel functions $J_0,
J_1$ only. \old{Therefore}\new{Consequently}, the recurrence relation
%
\begin{equation}
  J_2(kr) = \frac{2}{kr}J_1(kr) - J_0(kr)
  \label{eq:j2}
\end{equation}
%
is used, where $k$ and $r$ are the wavenumber and space-domain parameters,
respectively. Other differences are that empymod includes Fourier transforms,
hence time-domain calculation, and can model arbitrary rotated, finite bipoles.

The QWE and filter Hankel and Fourier transforms in \emph{transform} follow
\cite{GEO.12.Key}, with a few changes. The most important ones regarding speed
is vectorization, hence avoiding loops, and a splined version for the filter
method, in addition to the traditional lagged version: The lagged version
samples the wavenumber from the minimum to the maximum required wavenumber
given the required offsets and the chosen filter base with the spacing as
defined by the filter. It subsequently carries out the Hankel transform, and
interpolates for the required offsets in the space domain. The splined version,
on the other hand, uses a user-specified number of values per decade from the
minimum to the maximum wavenumber. It then interpolates for the required
wavenumber values, and does the Hankel transform afterwards. The lagged version
is \emph{very} fast. However, with the splined version a speed-up can be
achieved in comparison with the original FHT at higher precision if compared
with the lagged version. All filters published in the source codes of
\cite{GEO.09.Key, GEO.12.Key} are included in empymod, which includes Key's own
filters as well as the filters by \cite{TMS.82.Anderson} and by
\cite{GP.07.Kong}. (I refer throughout to paper to a specific filter with
author, year, and filter-size, for instance Key09-201.)

The most import part of \cite{GEO.12.Key} is the introduction of a new
quadrature algorithm to geophysics, named
qua\-dra\-ture-with-ex\-tra\-po\-la\-tion (QWE). QWE is a fast quadrature
method using the Shanks transformation \citep{JMP.55.Shanks} computed with
Wynn's epsilon algorithm \citep{MC.56.Wynn}. The advantage of quadrature over
filters is the ability to estimate the error. QWE continues until the absolute
error, estimated by the difference of subsequent iterations $n$, satisfies the
inequality
%
\begin{equation}
  |S^*_n-S^*_{n-1}| \le \varepsilon_r|S^*_n| + \varepsilon_a\ ,
  \label{eq:err}
\end{equation}
%
where $S^*$ is the extrapolated result, and $\varepsilon_r, \varepsilon_a$ are
the relative and absolute tolerance, respectively.

Whereas for the Hankel transform the three methods QWE, FHT, and QUAD, a
standard adaptive quadrature using QAGSE from the Fortran QUADPACK library, are
implemented, for the Fourier transform there are four methods implemented: QWE,
FHT with the sine and cosine filters, the standard FFT, and FFTLog
\citep{RAS.00.Hamilton}. The logarithmic fast Fourier transform FFTLog is
ideal for this operation, as generally a wide range of frequencies is required
to go from the frequency to the time domain. The FFTLog can yield faster
results than the QWE method and more precise than the FHT results, specifically
for land impulse responses at early times.

Calculation time can be reduced by using the splined version of the QWE and the
splined and lagged versions of the FHT. In addition, all time-consuming
calculations are set up to be able to run in parallel, with the help of the
Python-module \old{numexpr}\new{\texttt{numexpr}}. This can significantly speed
up calculations if you run big models with many layers and for many offsets and
frequencies. However, if you include empymod in an inversion scheme then it
might be better to parallelize the calls to empymod, instead of empymod itself.

It is important to note that calculations in the wavenumber domain depend
\emph{only} on offset $r = \sqrt{x^2+y^2}$; the scaling factor, which depends
on the angle $\varphi = \rm{atan2}(y, x)$, is multiplied afterwards. In order
to calculate for instance a circle around the source, the kernel has to be
called only once \new{for offset $r$}, and subsequently scaled by the
angle-dependent factor for each source-receiver pair. This makes the splined
version very powerful, as it can be used for irregularly distributed data%
\old{ as long as all receiver depths are the same.}%
\new{: you can calculate any amount of data points in the x-y-plane in one
single kernel call. This advantage fails as soon as the receivers have
different depths.}

The run-time comparisons tests were run on a Lenovo ThinkCentre running Ubuntu
16.04 64-bit, with 8\,GB of memory and an Intel Core i7-4770 CPU @ 3.40GHz x 8.
Comparing run times is always a difficult task, and between different languages
even more, here between Python (empymod), Fortran (EMmod, Dipole1D), and Matlab
(scripts from \cite{GEO.12.Key}). However, they do serve as comparison. I
used the Matlab-\texttt{timing} and Python-\texttt{timeit} functions, which
behave very similar. It is probably expected today, but still worth mentioning,
that the calculation is carried out in double precision.

\section{Comparison to analytical half-space solution}

\cite{PIER.10.Slob} published analytical frequency- and time-domain solutions
for the diffusive electric field in a VTI half-space, where source and receiver
can be placed anywhere in the half-space. As an example, I use the same model
as \cite{GEO.15.Hunziker} in their Figures~1 and~2: A half-space with
horizontal resistivity $\rho_\mathrm{h} = 1/3\,\Omega$\,m, anisotropy $\lambda
= \sqrt{10}$, and for frequency $f = 0.5\,$Hz. The source is located
horizontally at the origin ($x_\mathrm{s} = y_\mathrm{s} = 0\,$m) at a depth of
150\,m, and the depth of the receivers is 200\,m. The field is calculated on a
regular grid with a spacing of 10\,m. Figure~\ref{fig:analytical} shows the
analytical amplitude and phase results for the first quadrant (the other
quadrants are simply symmetric copies of it).
%
\plot*[btp]{analytical}{width=\textwidth}{Analytical solution for a half-space
  with $\rho_\mathrm{h} = 1/3\,\Omega$\,m, $\lambda = \sqrt{10}$, $f =
  0.5\,$Hz, $z_\mathrm{s} = 150\,$m, and $z_\mathrm{r} = 200\,$m, calculated on
  a regular grid with a spacing of 10 meters; (a) amplitude and (b) phase.}
%
The figure shows the result for x-directed electric source and receiver dipoles
($G^\mathrm{ee}_{xx}$), other configurations yield similar results (the choice
is based on the insight that this is the most interesting of all electric
source to electric receiver fields, as can be seen in \cite{GEO.15.Hunziker}
Figures~1 and~2).

The error of the amplitude is shown in Figure~\ref{fig:onederror-amplitude} for
different settings regarding the Hankel transform: (a) for a 51\,pt QWE with
relative tolerance $\varepsilon_r$ and absolute tolerance $\varepsilon_a$ of
$10^{-12}$ and $10^{-30}$, respectively; (b) for a 21\,pt QWE with
$\varepsilon_r = 10^{-8}$ and $\varepsilon_a = 10^{-30}$; (c) for a 15\,pt QWE
with $\varepsilon_r = 10^{-8}$ and $\varepsilon_a = 10^{-18}$; (d) using the
standard FHT method with the filter Key09-201; (e) using the splined FHT of the
same filter with 40 points per decade; and (f) using the lagged FHT of the same
filter.
%
\plot*[btp]{onederror-amplitude}{width=.92\textwidth}{Error levels for
  different Hankel transform settings, compared with the analytical solution in
  Figure~1. The high precision QWE (a) and the standard FHT (d) have very low
  error levels, generally in the order of $10^{-6}$\,\% or less. The lagged FHT
  has much higher error levels, and the effects of interpolation can be seen
  clearly in the ring-like structure. However, almost the entire error is below
  0.1\,\% ($10^{-1}$\,\%), only the dark black parts have an error of 1\,\% or
  slightly more. Hankel arguments for QWE: {[rel.\ tol., abs. tol., nr of
  pts]}; for splined FHT: {[pts/decade]}.}
%
The results from the high precision QWE (a) and the standard FHT (d) are almost
identical, even though they use completely different Hankel transforms. The
error is, in this case, not due to the method used for the Hankel transform,
but rather to the distinct differences between the analytical solution, which
uses the diffusive approximation, and the numerical code, which calculates the
complete field. If the QWE is calculated for lower tolerance levels, as shown
in (b) and (c), or the FHT is used with interpolation as in (e) and (f),
artefacts seem to arise from the Hankel transform and the interpolation. The
error of the phase is shown in Figure~\ref{fig:onederror-phase}, with the same
conclusion.
%
\plot*[btp]{onederror-phase}{width=.92\textwidth}{Same as
  Figure~\ref{fig:onederror-amplitude}, but for the phase. Interesting is to
  see that interpolation in the wavenumber domain (e) yields slightly different
  patterns than interpolation in space domain (f).}
%

Figure~\ref{fig:filtercomp-amplitude} shows a comparison of the error for all
nine included FHT-filters, for the amplitude (phase is not shown, but is very
similar). They are all very accurate, and by choosing an optimal filter one has
to decide between accuracy and speed, as shorter filters are faster. However,
we have to keep in mind that the accuracy of the filters depends on the model,
the offsets, and the frequencies. If we do not have an analytical solution, as
in this case, we cannot check the accuracy.
%
\plot*[btp]{filtercomp-amplitude}{width=.92\textwidth}{Comparison of the nine
  included FHT filters. All of them have an error mostly far below 1\,\%. For
  other models and other frequencies the result will be different. (Phase is
  not shown but looks very similar.)}
%

\new{Comparisons of empymod to other modelers using non-homogeneous models
follow in the ground-penetrating radar example and in the section
\emph{Comparison to Key (2009, 2012)}.}

\section{Comparison to Hunziker et al. (2015)}

\cite{GEO.15.Hunziker} derive the electromagnetic fields in astoundingly simple
equations for all 36 possible source-receiver combinations (electric and
magnetic sources and receivers in three directions $x,y,z$) by finding the
solution for the vertical electric field and then applying the duality
principle and reciprocity to derive all components. The corresponding code
EMmod is published on the SEG website, and in \cite{GEO.16.Hunziker} they
published with iEMmod an inversion routine for it.

EMmod is written in Fortran and C. On execution, all parameters are provided in
an input file, and the resulting responses are written to an output file. The
calculation is carried out for a regular grid in the space domain, with at
least two points in each direction. The code calculates the solution for all
wavenumbers for the first quadrant, carries out the wavenumber-to-space
transformation, and then copies the result for the other four quadrants,
writing everything to the output file. This approach works very well and even
for millions of cells. However, the usage is probably more academic. When it
comes to actual measurements one deals with dozens to at most hundreds of
offsets, and usually on an irregular grid. Dipole1D and empymod work more along
the practical approach.

Figure~\ref{fig:emmod-HS} shows the error of amplitude and phase for EMmod. On
the left side with a colorscale from $10^0$ to $10^2$\,\% as used in
\cite{GEO.15.Hunziker}, on the right side with a colorscale from $10^{-8}$ to
$10^0$\,\% as used in Figures~\ref{fig:onederror-amplitude}
and~\ref{fig:onederror-phase}. Note that the error calculation in
\cite{GEO.15.Hunziker} was done slightly different. Here I show the relative
error between the analytical result and the result from EMmod, where
\cite{GEO.15.Hunziker} shows the relative error between $\log_{10}$(analytical
result) and $\log_{10}$(result from EMmod).
%
\plot*[btp]{emmod-HS}{width=.7\textwidth}{Error of EMmod for the half-space
  model in Figure~1. On the left side with the same colorscale as in
  \cite{GEO.15.Hunziker}, on the right side with the colorscale as in Figures~2
  and~3.}
%
Figure~\ref{fig:emmod-HS} shows a few interesting points. The responses from
EMmod in this example have significantly less precision than the results from
empymod. This does not mean in any way that EMmod is less precise, as EMmod can
be adjusted to yield much preciser results. However, this would significantly
increase the run time, so it has to be kept in mind for the run time
comparison. The important point is that EMmod does \emph{always} do an
interpolation in the space-frequency domain, by default, a linear
interpolation. No interpolation is used in empymod with QWE or FHT as Hankel
transform, unless one specifies it to speed up the calculations (splined QWE or
splined and lagged FHT options), which will therefore yield preciser results.
This can be seen very nicely in the results. Figures~\ref{fig:emmod-HS} (b)
and (d) show white circles where the result is very precise. These are the
offsets close to those where the fields were actually calculated. The black
bands in-between are the areas where the result was interpolated. It also shows
that the spacing between calculated offsets increases with increasing offsets.
The error patterns from empymod in Figures~\ref{fig:onederror-amplitude} and
\ref{fig:onederror-phase} show in (a) and (d) a different pattern, displaying
differences between the analytical, diffusive solution and the numerical, full
wavefield solution. It is only for the splined and lagged versions that one
sees the same, circular patterns appearing, which are due to interpolation.

Table~\ref{tbl:analytical} shows run times for these models. A few notes worth
mentioning: empymod carries out a lot of input checks, as it is quite forgiving
in what you input. EMmod, on the other hand, reads the input file and writes
the result to an output file, and copies the first quadrant to the other three.
Both have \old{therefore}\new{then} some overhead, and the comparison is not
strictly 1:1. For the comparison the same model was used as shown above, on a
regular grid with a spacing of 100\,m. On the test-machine the standard QWE and
FHT empymod would run into memory issues for denser spacing, hence arrays of
more than some 10\,000s of offsets. The splined and lagged versions of EMmod
could handle it, however.
%
\tabl[btp]{analytical}{Run times for the half-space model shown in Figure~1, on
  a regular grid with spacing of 100\,m, for EMmod and different Hankel
  transform settings of empymod (times in milliseconds).}{
  \centering
  \begin{tabularx}{320pt}{rAAABBB}
  \toprule%
  EMmod & \multicolumn{3}{c}{QWE} & \multicolumn{3}{c}{FHT} \\
        & 1 & 2 & 3               & 1 & 2 & 3 \\
  \midrule%
   5040 & 10300 & 2530 & 1210 & 1480 & 875 & 6 \\
  \bottomrule%
  \end{tabularx}
}
%
QWE becomes faster for decreasing precision, not surprisingly, and the splined
and lagged version of FHT are faster than the standard version. The lagged FHT
is the fastest by quite a margin, and about 1/3 of that time is from the input
checks, so it takes only a few milliseconds to calculate the 11\,025 offsets.
What is interesting here is that the lagged FHT is still more precise than
EMmod with the given settings, which means a speed-up of a factor 1000 for the
same result.

Many 1D CSEM codes use the diffusive approximation that is valid for low
frequencies. The appealing part of the derivation of \cite{GEO.15.Hunziker} is
that it models the complete wavefield, hence it is also valid for high
frequencies and therefore wave-phenomena. As an extreme case,
\cite{GEO.15.Hunziker} show a 1D example for ground-penetrating radar with a
center frequency of 250\,MHz. To do so, they calculate 2048 frequencies in the
range of 1\,MHz to 2048\,MHz for the Fast Fourier transform from frequency to
time domain. Here I only calculated the FFT with 850 frequencies from 1\,MHz
to 850\,MHz and then zero-padded up to 2048\,MHz, for EMmod and empymod; tests
have shown that this is sufficient. Figure~\ref{fig:gpr} shows the model with
the survey setup and the four results for EMmod, \new{empymod with }QUAD for
relative and absolute tolerance of $10^{-12}$ and $10^{-20}$, respectively,
empymod with a 21\,pt QWE and relative and absolute tolerance of $10^{-10}$ and
$10^{-18}$, respectively, and empymod with FHT with the filter Key09-401,
together with the analytical solution for the arrival of the direct wave (red),
the wave refracted at the surface (yellow), and the wave reflected at the
subsurface interface (green).

EMmod and the QUAD and QWE transforms of empymod yield pretty much the same
result. It is only at later times that some noise can be seen, different in
each method. FHT shows the first arrivals well, however, afterwards the result
becomes extremely noisy.
%
\plot*[btp]{gpr}{width=1\textwidth}{GPR example for (b) EMmod, (c) QUAD
  for relative and absolute tolerance of $10^{-12}$ and $10^{-20}$,
  respectively, (d) empymod with 51\,pt QWE for relative and absolute tolerance
  of $10^{-8}$ and $10^{-15}$, respectively, and (e) empymod with FHT
  (Key09-401); run times are given in the subplot titles.}
%
Interesting are the calculation times for these three results. EMmod took
roughly 9.5 hours to calculate, empymod with QUAD about 7.8 hours, empymod with
QWE about 4.4 hours, and empymod with FHT under a minute. Note that QWE has
internally a check: If the kernel is too steep within given intervals, it uses
QUAD instead, as QWE cannot handle very steep functions. QWE uses in about 1/3
% (37%)
of the calculation of this example QUAD. Surprising is how good the FHT result
is, given that the filter was designed for CSEM data, hence frequencies 6 to 9
orders of magnitudes smaller and much bigger offsets. In any case,
EMmod/empymod are not optimized for nor intended to model GPR data, and
calculations in time-domain will be much faster and more precise. It is
nevertheless an interesting proof of concept and shows that EMmod/empymod model
the entire EM field. This leaves the question if it would be possible to
create a filter that works for the frequencies and the offsets required for GPR
calculations.

\new{Note that a common-midpoint (CMP) GPR-layout was chosen with increasing
offsets between the antennas, rather than the more common profiling layout with
fixed antenna separation, often with shielded antennas. The reason is that a
fixed offset calculation with shielded antennas of a 1D model would result in
exactly the same response at each measurement point, resulting in two flat
horizons, the direct and the reflected field. A CMP example is for this reason
more interesting than a profiling example.}


\section{Comparison to Key (2009, 2012)}

\cite{GEO.12.Key} not only introduces QWE to geophysics, but also compares QWE
to FHT for various filters, some of which were specifically created for the
comparison. In order to calculate the models he translates parts of Dipole1D
\citep{GEO.09.Key} from Fortran to Matlab. The models are therefore isotropic
and use the diffusive approximation. He summarizes succinct and clear the FHT
method, outlines the QWE method, and made the codes available from the SEG
website.

The inclusion of the filter method FHT and the quadrature method QWE in empymod
follows \cite{GEO.12.Key}, with one important exception: vectorization. The
Matlab code of \cite{GEO.12.Key} loops over frequencies, offsets, and
wavenumbers; the code was developed to compare the different hankel transforms,
not with speed in mind. In empymod both methods are vectorized, hence various
offsets, and in the case of FHT also various frequencies, can be calculated for
all required wavenumbers in one calculation, which changes the conclusion drawn
in \cite{GEO.12.Key} comparing the speed of QWE and FHT. The vectorized
approach is much faster but is limited by the memory of the computer. If the
model becomes too big, which depends on numbers of layers, frequencies,
offsets, and wavenumbers, the calculation has to be carried out by looping over
frequencies or offsets or both; the code never loops over wavenumbers.

In \cite{GEO.12.Key}, the standard QWE is always faster than the standard FHT.
In the vectorized version of empymod, the standard QWE \emph{can} be faster
than the standard FHT if the model is big and many offsets are required, but
often it is slower. Furthermore, the lagged FHT is generally much faster than
the QWE method, splined or not.

Table~\ref{tbl:runtimes} lists the run times for exactly the same models as
\cite{GEO.12.Key} compared in his Table~1: 1\,km water layer, followed by a
1\,km overburden of 1\,$\Omega\,$m, 100\,m reservoir of 100\,$\Omega\,$m, and
lastly 1 or 96 layers of underburden with a resistivity of 1\,$\Omega\,$m;
frequency is 1\,Hz, source depth 990\,m, and receivers are on the sea-bottom;
there are 1, 5, 21, 81, or 321 offsets between 0.5 and 20\,km. In this
comparison, a 9\,pt QWE is compared to a 201\,pt (Key12-201) and a 801\,pt
filter (Anderson82-801), where the absolute and relative tolerance are set so
that the QWE achieves a similar error-level as the filters ($\varepsilon_r =
10^{-6}$ for the standard QWE and $10^{-2}$ for the splined version,
$\varepsilon_a = 10^{-24}$ in both cases). In order to make a fair comparison I
re-run the script from \cite{GEO.12.Key}, and the run times on the test-machine
are roughly twice as fast as in the original paper. Next to it are the results
from empymod. It can be seen from the results that empymod is significantly
faster.%
%
\tabl[btp]{runtimes}%
{Run times comparing the codes from \cite{GEO.12.Key} (Key) and empymod (Wer)
for the same cases as in \cite{GEO.12.Key}, Table~1 (in milliseconds).}{
  \centering
\begin{tabularx}{1.05\textwidth}{ccYYWWYYWWYYWW}
  \toprule
  %
  & & \multicolumn{4}{c}{QWE: 9\,pt}&\multicolumn{4}{c}{FHT: 201\,pt filter}&
  \multicolumn{4}{c}{FHT: 801\,pt filter} \\
  %
  \cmidrule(rl){3-6} \cmidrule(rl){7-10} \cmidrule(rl){11-14}
  %
  & &
  \multicolumn{2}{>{\columncolor{MyGray}\arraybackslash}c}{Normal} &
  \multicolumn{2}{c}{Splined} &
  \multicolumn{2}{>{\columncolor{MyGray}\arraybackslash}c}{Normal} &
  \multicolumn{2}{c}{Lagged} &
  \multicolumn{2}{>{\columncolor{MyGray}\arraybackslash}c}{Normal} &
  \multicolumn{2}{c}{Lagged} \\
  %
  Lay& Off& Wer& Key& Wer& Key& Wer&  Key& Wer& Key& Wer&  Key& Wer& Key\\
  \midrule
  %
    5&   1&   7&  16&   2&  28&   1&   19&   1&  20&   2&   76&   2&  76\\
    5&   5&  13&  69&   4&  29&   3&   92&   1&  24&   7&  365&   3&  82\\
    5&  21&  19& 300&   6&  36&   8&  384&   2&  25&  25& 1522&   3&  83\\
    5&  81&  37&1154&  16&  62&  30& 1475&   2&  28& 100& 5865&   3&  86\\
    5& 321& 108&4581&  58& 165& 124& 5867&   2&  41& 437&23165&   3&  99\\
  \midrule
  100&  21& 118& 463&  13&  51&  92&  604&   8&  37& 304& 2402&  19& 127\\
  100&  81& 296&1792&  23&  75& 337& 2313&   8&  40&1234& 9215&  19& 129\\
  100& 321&1021&7226&  65& 178&1408& 9350&  10&  53&5448&36520&  19& 142\\
  %
  \bottomrule
\end{tabularx}}%
%
Important to note is, however, not the absolute speed of empymod, but the
difference between the various Hankel transform methods. The standard QWE
method, for a similar error level as FHT, is faster than the standard FHT mode
only for many offsets. More importantly, the lagged FHT is generally much
faster than any QWE. QWE is very useful to check the error level of a result.
If many kernel evaluations have to be carried out, and speed is of importance,
the lagged FHT is the preferred method. As \cite{GEO.12.Key} loops over each
wavenumber, the difference between the vectorized and non-vectorized version
can be best seen in the long 801\,pt filter.

These results regarding speed of calculation do not change the fact that the
advantage of the QWE method is to have an estimate of the error. However, the
filters designed for CSEM problems seem to be very good at solving them, as can
be seen in the low error levels in Figures~\ref{fig:onederror-amplitude}
and~\ref{fig:onederror-phase} or in \cite{GEO.12.Key}.


Table~\ref{tbl:runtimes2} lists the run times for the same model as before, but
for empymod and Dipole1D using the FHT method with the filter Key09-201, the
default filter in both Dipole1D and empymod. For the regular versions empymod
and Dipole1D perform very similar; empymod is a little faster for the smaller
tests, Dipole1D is slightly faster for the bigger tests. It can be seen from
the results that the parallel optimisation in empymod can result in a
significant speed-up. If the lagged convolution optimisation is chosen,
empymod is significantly faster than Dipole1D. In empymod, additional offsets
come at basically no cost in the lagged version, whereas Dipole1D still uses
more time to calculate more offsets. This is most likely due to the writing of
the output file, that becomes bigger with bigger offsets.
%
\tabl[btp]{runtimes2}
{Run times comparing Dipole1D \citep{GEO.09.Key} and empymod, using the filter
Key09-201 (times in milliseconds). Optimisation: [-] None, [par] parallel,
[spl] lagged FHT.}{ \centering
  \begin{tabularx}{.5\textwidth}{ccAAABB}
  \toprule
  %
     &    & \multicolumn{3}{c}{empymod} & \multicolumn{2}{c}{Dipole1D}\\
  %
  \cmidrule(rl){3-5} \cmidrule(rl){6-7}
  %
  Lay& Off&  - & par & spl &   - & spl\\
  \midrule
  %
    5&   1&    1&   2& 1&    4&  4\\
    5&   5&    3&   3& 2&    6&  5\\
    5&  21&    9&   5& 2&   12&  8\\
    5&  81&   30&  10& 2&   36& 18\\
    5& 321&  124&  34& 2&  130& 56\\
  \midrule
  100&  21&   91&  53& 9&   90& 13\\
  100&  81&  340& 106& 9&  333& 23\\
  100& 321& 1423& 313& 9& 1321& 62\\
  %
  \bottomrule
\end{tabularx}}%
%
Dipole1D was compiled using the free and open-source \emph{gfortran} compiler.
Compiling it with the (proprietary) \emph{ifort} compiler might result in
slightly faster run times. Note that Dipole1D always calculates all six
receiver components, whereas empymod only calculates the required component(s).
On the other hand empymod computes the whole, anisotropic wavefield, whereas
Dipole1D is isotropic and uses the diffusive approximation.

\section{Comparison of different Fourier transforms}

The following is a comparison of different Fourier transforms with the
analytical solution. The electromagnetic impulse response of an isotropic
half-space model below air with interface at $z=0\,$m for an x-directed
electromagnetic source at the origin
($x_\textrm{s}=y_\textrm{s}=z_\textrm{s}=0\,$m) and a receiver at an inline
offset of 6\,km at the surface ($r = x_\textrm{r}=6\,$km,
$y_\textrm{r}=z_\textrm{r}=0\,$m) is given by
%
\begin{equation}
  E_x(\rho,r,t) = \frac{1}{8} \sqrt{\frac{\mu_0^3}{\pi^3 t^5 \rho}}
                    \exp\left(-\frac{\mu_0 r^2}{4\rho t}\right)\ ,
  \label{eq:impulse}
\end{equation}
%
where $\rho$ is the halfspace resistivity, $\mu_0$ is the permeability of free
space, and $t$ is time (neglecting the impulse of the airwave at $t=0\,$s), as
given by \citet[][ eq. 5.38]{PhD.97.Wilson}.

Figure~\ref{fig:impulse} shows the result of this comparison, where FFTLog with
6\,ms is the fastest method and FFT with 562\,ms the slowest one. Interesting
is to see how many frequencies are used internally, as listed in
Table~\ref{tbl:time}.
%
\tabl[btp]{time}{Run-times for the four different Fourier transforms, and
  frequency range they require. More frequencies does not necessarily mean
  faster nor more precise. All Fourier transform do interpolation either in
  frequency or in time domain.}{
  \centering
  \begin{tabular}{lcccc}
  \toprule
  %
   Method  &  \# freq   & min freq & max freq & time \\
           &            & Hz       & Hz       & ms \\
  %
  \midrule
  %
    FFTLog     &  70& 1.8e-5& 1.4e2 &   6\\
    Sine-filter & 128& 5.3e-7& 5.7e3 &  11\\
    QWE        & 178& 7.9e-5& 6.3e4 & 298\\
    FFT        &  61& 5.0e-5& 5.2e1 & 562\\
  %
  \bottomrule
\end{tabular}}%
%
Even though FFT uses the least frequencies it is the slowest (and least
precise) method. The reason is that it only calculates the response for 61
frequencies, but for the actual FFT it afterwards interpolates them at $2^{20}
= 1\,048\,579$ frequencies to have a regular array from 5e-5\,Hz to 52\,Hz. To
get a better result we would have to get to lower and higher frequencies, but
then the number of required frequencies would even grow bigger. Note that this
example is a difficult one, with source and receiver at the interface between
air and the subsurface. Models where source and receiver are not at the same
depth and not at the air-interface, for instance in marine CSEM where source
and receiver are in the water layer, are generally much easier, in other words,
faster and more precise.
%
\plot*[btp]{impulse}{width=\textwidth}{Comparison of different Fourier
  transforms for an impulse response at the interface between air and a
  halfspace. FFTLog is the fastest, and FFT the slowest in this example.}
%

\section{Conclusions}

The presented code empymod is an open-source electromagnetic modeler that can
model the full wavefield of a 3D source in a VTI layered-earth. Three different
Hankel transform methods (quadrature, quadrature-with-extrapolation, and
filters) are included into empymod, as well as four different Fourier transform
methods (FFT, FFTLog, quadrature-with-extrapolation, and sine/cosine-filters).
Additional to the obvious application of modeling and inversion of
electromagnetic data, empymod can therefore be used to investigate into the
differences between different Hankel and Fourier transforms, or between
different filters for full-wavefield electromagnetic calculations over a wide
range of frequencies as well as times. The examples show that empymod can
compete with existing code in both speed and accuracy, even though it is
written in a dynamically typed language.

Outside of the traditional scope empymod can be very educative for someone who
is just starting to get interested in electromagnetic modeling or even just in
Hankel and Fourier transforms, hence for educational purpose: it is well suited
for students, specifically as Python is becoming more and more widespread in
academia, and the number of Universities that teach geophysicist Python (mostly
instead of Matlab) is growing annually. Python, like Matlab, has the advantage
of very fast developing times, and considerably lower entry barriers than C or
Fortran for beginners. \new{And, unlike Matlab, Python is completely free and
open-source. The module empymod is accordingly ideal for truly reproducible
research.}

There are many possibilities to improve empymod or to add additional
functionalities, for instance more modeling routines such as loop sources.
Improving code abstraction or creating a graphical user interface are further
possibilities. The code comes with a complete testing-suite, but the included
benchmarks could also be improved and extended with regression tests.


\section{Acknowledgment}

I would like to thank the \emph{Consejo Nacional de Ciencia y Tecnología},
México (CONACYT) for funding this postdoc, the \emph{Instituto Mexicano del
Petróleo} (IMP) for allowing me to publish the code under a free software
license, and the entire electromagnetic research group of Aleksandr Mousatov at
the IMP for fruitful discussions. I owe a special thanks to Jürg Hunziker for
answering all my questions regarding his code and publication. I thank the
assistant editor, C. Torres-Verdin, the associate editor, F. Broggini, and Jan
Thorbecke, Vladimir Puzyrev, and a third, anonymous reviewer, as well as Jürg
Hunziker and Kerry Key, whose feedback greatly improved the clarity of the
manuscript.

\append[apd:code]{Code structure and basic usage}

\new{
Figure~\ref{fig:codestructure} shows the structure of empymod.%
}
%
\plot*[btp]{codestructure}{}{Code structure of the module empymod showing the
  different files and the main modeling routines \texttt{bipole},
  \texttt{dipole}, and \texttt{analytical}.}%
%
\new{
The important routines for a user who only wants to use the code and is not
interested in the details are located in \texttt{model.py}. The three main
modeling routines are \texttt{bipole}, \texttt{dipole}, and
\texttt{analytical}. With \texttt{bipole} you can calculate frequency- and
time-domain responses for arbitrary oriented, electric or magnetic bipole
sources and receivers of finite length. The routine \texttt{dipole} is
basically a subset of \texttt{bipole}, and is a good starting point if you
want to create your own modeling routine. And \texttt{analytical} is an
interface to the analytical full- and halfspace solutions that come with
empymod. In addition there are the routines \texttt{wavenumber} to calculate
the wavenumber-frequency result of a dipole, and \texttt{gpr} to calculate the
response for a ground-penetrating radar with a given center frequency. More
modeling routines will hopefully be added in the future.
}

\new{
The file \texttt{kernel.py} contains the wavenumber-frequency domain
calculation as given in \cite{GEO.16.Hunziker}. It also contains analytical
solutions, namely the complete space-frequency domain fullspace solution as
given in \cite{GEO.16.Hunziker}, and the space-frequency and space-time domain
diffusive halfspace solutions for electric sources and receivers as given in
\cite{PIER.10.Slob}. All calculations related to the Hankel and Fourier
transforms are carried out in the file \texttt{transform.py}.  Digital filters
are collected in \texttt{filter.py}. It contains, as of now, 9 filters for the
Hankel transform \citep{TMS.82.Anderson, GEO.12.Key, GEO.09.Key, GP.07.Kong},
and 5 filters for the Fourier transform \citep{GEO.12.Key, GEO.09.Key}.
}

\new{
The modeling routines rely heavily on \texttt{utils.py}. In this routine are
the input tests and other checks, such as to get the layer number in which
source and receivers reside, and everything related to the verbosity of the
code.  And \texttt{\_\_init\_\_.py} is a required file, such that Python
recognises empymod as a module that can be loaded. In the case of empymod it
does three more things, common to Python init-files: It provides the version
number, it loads the most important routine (\texttt{bipole}, \texttt{dipole},
\texttt{analytical}) so they are available from the module itself, and it
contains some documentation.
}

\new{
As an exemplary minimal example we calculate a halfspace-model with
air/halfspace interface at $0\,$m; electric, $x$-directed source at
$x_s=y_s=0\,$m and $z_s=1\,$m; electric receiver at $x_r=1000\,$m, $y_r=0\,$m,
$z_r=1\,$m, with azimuth $\theta_r=45^\circ$ and dip $\varphi_r=10^\circ$; and
frequency of $2\,$Hz.
}

\begin{verbatim}
>>> from empymod import bipole
>>> em = bipole(
>>>     src = [0, 0, 1, 0, 0],
>>>     rec = [1000, 0, 1, 45, 10],
>>>     depth = 0,         # interface
>>>     res = [2e14, 100], # air, halfspace
>>>     freqtime = 2,      # frequency
>>>     verb = 1)          # verbosity
>>> print('%0.4e, %0.4ei' % (em.real, em.imag))
2.2042e-08, -7.1539e-10i
\end{verbatim}

\new{The function \texttt{bipole} takes many more input arguments, but these
five are mandatory. You can find more extensive examples under
\url{https://empymod.github.io} or in the Geophysical Tutorial of
\cite{TLE.17.Werthmuller}.}

% REFERENCES
\bibliographystyle{seg}
\begin{thebibliography}{}
\itemsep0pt

\bibitem[Anderson, 1975]{TRP.75.Anderson}
Anderson, W.~L., 1975, Improved digital filters for evaluating {F}ourier and
  {H}ankel transform integrals: Technical report, U.S. Geological Survey.
\newblock
  (\href{https://pubs.er.usgs.gov/publication/70045426}{https://pubs.er.usgs.gov/publication/70045426}).

\bibitem[Anderson, 1979]{GEO.79.Anderson}
--------, 1979, Numerical integration of related {H}ankel transforms of orders
  0 and 1 by adaptive digital filtering: Geophysics, {\bf 44}, 1287--1305.
\newblock (\href{http://dx.doi.org/10.1190/1.1441007}{doi: 10.1190/1.1441007}).

\bibitem[Anderson, 1982]{TMS.82.Anderson}
--------, 1982, Fast {H}ankel transforms using related and lagged convolutions:
  ACM Trans. Math. Softw., {\bf 8}, 344--368.
\newblock (\href{http://doi.acm.org/10.1145/356012.356014}{doi:
  10.1145/356012.356014}).

\bibitem[Anderson, 1984]{GEO.84.Anderson}
--------, 1984, {On: “Numerical integration of related Hankel transforms by
  quadrature and continued fraction expansion” by \cite{GEO.83.Chave}}:
  Geophysics, {\bf 49}, 1811--1812.
\newblock (\href{http://dx.doi.org/10.1190/1.1441595}{doi: 10.1190/1.1441595}).

\bibitem[Anderson, 1989]{GEO.89.Anderson}
--------, 1989, A hybrid fast {H}ankel transform algorithm for electromagnetic
  modeling: Geophysics, {\bf 54}, 263--266.
\newblock (\href{http://dx.doi.org/10.1190/1.1442650}{doi: 10.1190/1.1442650}).

\bibitem[Chave, 1983]{GEO.83.Chave}
Chave, A.~D., 1983, Numerical integration of related {H}ankel transforms by
  quadrature and continued fraction expansion: Geophysics, {\bf 48},
  1671--1686.
\newblock (\href{http://dx.doi.org/10.1190/1.1441448}{doi: 10.1190/1.1441448}).

\bibitem[Chave, 2009]{GJI.09.Chave}
--------, 2009, On the electromagnetic fields produced by marine frequency
  domain controlled sources: Geophysical Journal International, {\bf 179},
  1429--1457.
\newblock (\href{http://dx.doi.org/10.1111/j.1365-246X.2009.04367.x}{doi:
  10.1111/j.1365-246X.2009.04367.x}).

\bibitem[Chave et~al., 1991]{B.SEG.91.Chave}
Chave, A.~D., S.~C. Constable, and R.~N. Edwards, 1991, Electrical exploration
  methods for the seafloor, {\it in} Electromagnetic Methods In Applied
  Geophysics Vol.\ 2: SEG, Investigations in Geophysics, No.~3, 12, 931--966.
\newblock (\href{http://dx.doi.org/10.1190/1.9781560802686}{doi:
  10.1190/1.9781560802686}).

\bibitem[Edwards, 2005]{SG.05.Edwards}
Edwards, R.~N., 2005, Marine controlled source electromagnetics: principles,
  methodologies, future commercial applications: Surveys in Geophysics, {\bf
  26}, 675--700.
\newblock (\href{http://dx.doi.org/10.1007/s10712-005-1830-3}{doi:
  10.1007/s10712-005-1830-3}).

\bibitem[Ghosh, 1971]{GP.71.Gosh}
Ghosh, D.~P., 1971, The application of linear filter theory to the direct
  interpretation of geoelectrical resistivity sounding measurements:
  Geophysical Prospecting, {\bf 19}, 192--217.
\newblock (\href{http://dx.doi.org/10.1111/j.1365-2478.1971.tb00593.x}{doi:
  10.1111/j.1365-2478.1971.tb00593.x}).

\bibitem[Hamilton, 2000]{RAS.00.Hamilton}
Hamilton, A. J.~S., 2000, Uncorrelated modes of the non-linear power spectrum:
  Monthly Notices of the Royal Astronomical Society, {\bf 312}, 257--284.
\newblock (\href{http://dx.doi.org/10.1046/j.1365-8711.2000.03071.x}{doi:
  10.1046/j.1365-8711.2000.03071.x}).

\bibitem[Hunziker et~al., 2015]{GEO.15.Hunziker}
Hunziker, J., J. Thorbecke, and E. Slob, 2015, The electromagnetic response in
  a layered vertical transverse isotropic medium: {A} new look at an old
  problem: Geophysics, {\bf 80}, F1--F18.
\newblock (\href{http://dx.doi.org/10.1190/geo2013-0411.1}{doi:
  10.1190/geo2013-0411.1}).

\bibitem[Hunziker et~al., 2016]{GEO.16.Hunziker}
Hunziker, J., J. Thorbecke, J. Brackenhoff, and E. Slob, 2016, Inversion of
  controlled-source electromagnetic reflection responses: Geophysics, {\bf 81},
  F49--F57.
\newblock (\href{http://dx.doi.org/10.1190/geo2015-0320.1}{doi:
  10.1190/geo2015-0320.1}).

\bibitem[Key, 2009]{GEO.09.Key}
Key, K., 2009, {1D} inversion of multicomponent, multifrequency marine {CSEM}
  data: {M}ethodology and synthetic studies for resolving thin resistive
  layers: Geophysics, {\bf 74}, F9--F20.
\newblock (\href{http://dx.doi.org/10.1190/1.3058434}{doi: 10.1190/1.3058434}).

\bibitem[Key, 2012]{GEO.12.Key}
--------, 2012, Is the fast {H}ankel transform faster than quadrature?:
  Geophysics, {\bf 77}, F21--F30.
\newblock (\href{http://dx.doi.org/10.1190/GEO2011-0237.1}{doi:
  10.1190/GEO2011-0237.1}).

\bibitem[Kong, 2007]{GP.07.Kong}
Kong, F.~N., 2007, Hankel transform filters for dipole antenna radiation in a
  conductive medium: Geophysical Prospecting, {\bf 55}, 83--89.
\newblock (\href{http://dx.doi.org/10.1111/j.1365-2478.2006.00585.x}{doi:
  10.1111/j.1365-2478.2006.00585.x}).

\bibitem[Løseth and Ursin, 2007]{GJI.07.Loseth}
Løseth, L.~O., and B. Ursin, 2007, Electromagnetic fields in planarly layered
  anisotropic media: Geophysical Journal International, {\bf 170}, 44--80.
\newblock (\href{http://dx.doi.org/10.1111/j.1365-246X.2007.03390.x}{doi:
  10.1111/j.1365-246X.2007.03390.x}).

\bibitem[Nekut and Spies, 1989]{PIEEE.89.Nekut}
Nekut, A.~G., and B.~R. Spies, 1989, Petroleum exploration using
  controlled-source electromagnetic methods: Proceedings of the IEEE, {\bf 77},
  338--362.
\newblock
  (\href{http://ieeexplore.ieee.org/lpdocs/epic03/wrapper.htm?arnumber=18630}{doi:
  10.1109/5.18630}).

\bibitem[Shanks, 1955]{JMP.55.Shanks}
Shanks, D., 1955, Non-linear transformations of divergent and slowly
  convergent sequences: Journal of Mathematics and Physics, {\bf 34}, 1--42.
\newblock (\href{http://dx.doi.org/10.1002/sapm19553411}{doi:
  10.1002/sapm19553411}).

\bibitem[Slob et~al., 2010]{PIER.10.Slob}
Slob, E., J. Hunziker, and W.~A. Mulder, 2010, Green's tensors for the
  diffusive electric field in a {VTI} half-space: PIER, {\bf 107}, 1--20.
\newblock (\href{http://dx.doi.org/10.2528/PIER10052807}{doi:
  10.2528/PIER10052807}).

\bibitem[Wait, 1982]{B.AP.82.Wait}
Wait, J.~R., 1982, Geo-{E}lectromagnetism: Academic Press Inc.
\newblock ({I}SBN: 978-0127308807).

\bibitem[Werthmüller, 2017]{TLE.17.Werthmuller}
Werthmüller, D., 2017, Getting started with controlled-source electromagnetic
1D modeling: The Leading Edge, {\bf 36}, 352--355.
\newblock (\href{http://dx.doi.org/10.1190/tle36040352.1}{doi:
  10.1190/tle36040352.1}).

\bibitem[Wilson, 1997]{PhD.97.Wilson}
Wilson, A.~J.~S., 1997, The equivalent wavefield concept in multichannel
transient electromagnetic surveying: Ph.D., University Of Edinburgh.
\newblock (\href{http://hdl.handle.net/1842/7101}{uri:
hdl.handle.net/1842/7101}).

\bibitem[Wynn, 1956]{MC.56.Wynn}
Wynn, P., 1956, {On a device for computing the $e_m(S_n)$ tranformation}:
  Math. Comput., {\bf 10}, 91--96.
\newblock (\href{http://dx.doi.org/10.1090/S0025-5718-1956-0084056-6}{doi:
  10.1090/S0025-5718-1956-0084056-6}).

\bibitem[Ziolkowski and Wright, 2012]{IEEE.12.Ziolkowksi}
Ziolkowski, A., and D. Wright, 2012, The potential of the controlled source
  electromagnetic method: A powerful tool for hydrocarbon exploration,
  appraisal, and reservoir characterization: Signal Processing Magazine, IEEE,
  {\bf 29}, 36--52.
\newblock (\href{http://dx.doi.org/10.1109/MSP.2012.2192529}{doi:
  10.1109/MSP.2012.2192529}).

\end{thebibliography}

\end{document}
