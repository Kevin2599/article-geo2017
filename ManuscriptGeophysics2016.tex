% \documentclass[manuscript]{geophysics}
% \documentclass[paper,twocolumn,twoside]{geophysics}
% \documentclass[paper]{geophysics}
\documentclass[manuscript,revised]{geophysics}
%%fakesection ===    PACKAGES & DEFINITIONS    ===

% \usepackage{draftwatermark}
% \SetWatermarkLightness{.95}
% \SetWatermarkFontSize{2cm}
% % \SetWatermarkText{\shortstack[c]{Submitted to \texttt{Geophysics}\\23 Oct 2012}}
% \SetWatermarkText{\shortstack[c]{Re-submitted to \texttt{Geophysics}\\05 Feb 2013}}

%% ~ ADDITIONAL PACKAGES TO GEOPHYSICS.CLS
\DeclareGraphicsExtensions{.pdf,.png,.jpg}
\usepackage{color}
\usepackage{tabularx}
\usepackage{colortbl}
\usepackage{booktabs}
\usepackage[USenglish]{babel}
\usepackage[utf8]{inputenc}
\usepackage{lmodern}
\usepackage[T1]{fontenc}
\usepackage{amssymb, amsmath, amsfonts}
% \usepackage[pdftex, draft]{hyperref}
\usepackage{xspace}                 % For spaces after \newcommand-strings
\usepackage{upquote}

% vvv STUFF NOT FOR SUBMISSION
\usepackage[pdftex, final]{hyperref}
\definecolor{myblue}{rgb}{0,0,.5}
\hypersetup{allcolors=myblue, allbordercolors={0 0 .5},
            pdfborderstyle={/S/U/W .5}, colorlinks=true,}
% ^^^ STUFF NOT FOR SUBMISSION

% Figure directory
\renewcommand{\figdir}{figures}
\usepackage[strings]{underscore}

%
% Figure and table widths
%   SEGTex               column width = 250 pt
%   Geophysics           column width = 240 pt (3.33 in)
%   Geophysics       1.3 column width = 312 pt ( 4.33 in)
% \plot[btp]{figure1}{width=240pt}{Diff porosities higher than roughly 40~\%.}
% \plot*[btp]{figure11}{width=\textwidth}{(a) i horizontal resistivities.}
%
\definecolor{MyGray}{gray}{0.85}
\newcolumntype{Y}{>{\columncolor{MyGray}\raggedleft\arraybackslash}r}
\newcolumntype{W}{>{\raggedleft\arraybackslash}r}
\newcolumntype{A}{>{\columncolor{MyGray}\raggedleft\arraybackslash}X}
\newcolumntype{B}{>{\raggedleft\arraybackslash}X}

\hyphenation{iso-tro-pic pe-ne-tra-ting}

\begin{document}

\title{An open-source \new{1D }electromagnetic modeler in Python: empymod}

\renewcommand{\thefootnote}{1}% \fnsymbol{footnote}}

\ms{GEO-2016-0626}

\address{Instituto Mexicano del Petróleo,
         Eje Central Lázaro Cárdenas Norte 152,
         Col. San Bartolo Atepehuacan C.P. 07730,
         Ciudad de México, México.
         E-mail: \href{mailto:dieter@werthmuller.org}{Dieter@Werthmuller.org}.}

\author{Dieter Werthmüller\footnotemark[1]}

\footer{}
\lefthead{Werthmüller}
\righthead{Open-source \new{1D} EM modeler in Python}

\maketitle

%%fakesection ===    ABSTRACT    ===
\begin{abstract} % 1-2 sentence(s) each
\old{
  The free software empymod combines two earlier presented algorithms in this
  journal, creating a new code written in Python that is faster and leaner.
  The main objective is to present a three-dimensional layered-earth
  electromagnetic modeler with vertical transverse isotropy in an easy
  accessible programming language, both in terms of costs and learning curve,
  and in a collaborative way. The code is hosted in a web-based Git repository
  under a lax permissive license, allowing anyone to use it, even for
  commercial purposes, as well as to contribute to its development.
  Comparisons show that this code is as precise and faster than the codes it is
  based upon, thanks to the use of different Hankel transforms and the
  throughout calculation.  This code might certainly be useful for
  professionals in the electromagnetic area, but I specifically hope this code
  to be useful for educational purposes.
} % Note: replace:
  % ``vectorization of the calculation'' with ``calculation as arrays''.
%
\new{%
  The presented  Python-code empymod computes the three-dimensional
  electromagnetic field in a layered-earth with vertical transverse isotropy
  by combining and extending two earlier presented algorithms in this journal.
  The bottleneck in frequency- and time-domain calculations of electromagnetic
  responses derived in the wavenumber-frequency domain is the transformations
  from wavenumber to space domain and from frequency to time domain, the
  so-called Hankel and Fourier transforms.  Three different Hankel transform
  methods (quadrature, quadrature-with-extrapolation, and filters) and four
  different Fourier transform methods (FFT, FFTLog,
  quadrature-with-extrapolation, and filters) are included in empymod, which
  allows to compare these different methods in terms of speed and precision.
  The best transform in terms of speed and precision depends on the modeled
  frequencies: available filters, for instance, are very fast and precise for
  frequencies in the range of controlled-source electromagnetic data, but fail
  in the frequency range of ground penetrating radar. Conventional quadrature,
  on the other hand, is in comparison very slow but can model any frequency.
  Examples comparing empymod with analytical solutions and with existing
  electromagnetic modelers illustrate the capabilities of empymod.
}
\end{abstract}

\section{Introduction}

% 1.1 CSEM in hydrocarbon exploration
The potential of electromagnetic methods for the detection of hydrocarbon
reservoirs is known for some decades, see for instance \cite{PIEEE.89.Nekut} or
\cite{B.SEG.91.Chave}. More recent, good overviews of the methodology and its
applications are give by \cite{SG.05.Edwards} and \cite{IEEE.12.Ziolkowksi}.
Whereas in the early days everything happened in idealized, isotropic one
dimensional (1D) earth models, it is generally agreed that the often complex
geology of hydrocarbon reservoirs requires two dimensional (2D) and three
dimensional (3D), anisotropic forward modelers. However, 1D models are still
very important\new{.} \old{, not last because they are very fast. Many problems
can be simplified to and henceforth solved with 1D models. More importantly, 1D
allows} \new{Besides that their calculation is very fast they allow} to study
single, isolated effects to the electromagnetic field, which is a crucial
foundation in understanding the electromagnetic field behaviour and a necessity
for understanding the phenomena at higher dimensions. 1D models are furthermore
often used in inversion routines of higher dimensions, for instance to generate
a starting model or \new{to calculate the primary fields for 2D/3D modeling and
inversion routines.} \old{embed a 3D body in a 1D background.}

% 1.2 Existing 1D solutions
Solutions for the electromagnetic fields in a layered-earth model have been
solved and published extensively using different approaches. The importance and
wide\-spread use of 1D models is shown in the continuous stream of publications
in this area, even in recent years. \cite{GJI.07.Loseth} solved the problem
with the scattering matrix formulation for a 1D earth with general anisotropy,
spanning the frequency range from controlled-source electromagnetics (CSEM) to
ground-pe\-ne\-tra\-ting radar (GPR).  \cite{GJI.09.Chave} presented a solution
in terms of independent and unique transverse electric (TE) and tangential
magnetic (TM) modes for the electrical isotropic case using the diffusive
approximation (without displacement currents, valid for low frequencies such as
in CSEM). \cite{GEO.09.Key} demonstrated why 1D models still matter by testing,
for instance, the benefit of additional frequencies in an inversion
routine\old{. Key}\new{; he} follows and extends the magnetic vector potential
approach by \cite{B.AP.82.Wait} for forward modeling, using the isotropic,
diffusive low-frequency approach.  \cite{GEO.15.Hunziker} obtained the
electromagnetic field in a layered earth with vertical transverse iso\-tro\-py
(VTI) by solving two equivalent scalar equations with a scalar global
reflection coefficient.  All of the four citations regarding 1D solutions have
quite extensive reference lists about the history of 1D forward modeling, the
last one featuring an interesting review of the history of 1D electromagnetic
derivations spanning almost 200 years.

% 1.3 Hankel transform
A crucial as well as very interesting part of electromagnetic modeling is,
once one moves from the theoretical derivation to the numerical implementation,
the Hankel transform involved in the transformation from the
wave\-num\-ber-fre\-que\-ncy domain to the space-frequency domain. 
\new{Because what is common in all these approaches is that they derive
solutions in the wavenumber-frequency domain. This in turn requires a Hankel
transform, which is a numerically expensive, infinite integral containing
oscillating, slowly decaying Bessel functions (the Hankel transform is also
known as the Fourier-Bessel transform).} The use of digital filters is quite
common in geophysics, known as the fast Hankel transform method (FHT), as
introduced by \cite{GP.71.Gosh}, and popularized by \cite{TRP.75.Anderson,
GEO.79.Anderson, TMS.82.Anderson} thanks to his freely available Fortran
routines. Naturally, standard quadrature can be used as well for the Hankel
transform, see for instance \cite{GEO.83.Chave}, and hybrid routines using both
methods were published by \cite{GEO.84.Anderson, GEO.89.Anderson}. The topic
got picked up again recently with new filters being published by
\cite{GP.07.Kong} and \cite{GEO.09.Key, GEO.12.Key}

% 1.4 1D codes and licenses
In terms of freely available and open-source code there are a few examples.
\cite{GEO.09.Key} published his forward modeling and inversion code Dipole1D,
written in Fortran. The dipole can be placed anywhere in the stack of layers
and can have arbitrary orientation and dip. Dipole1D computes the electric and
magnetic fields to an electric source, using the FHT method. \cite{GEO.12.Key}
published additional code with his introduction of the
qua\-dra\-ture-with-ex\-tra\-po\-la\-tion method (QWE) and its comparison to
the FHT method, for which he translated some of the Dipole1D-Fortran code to
Matlab. \cite{GEO.15.Hunziker} published their code EMmod (Fortran and C),
in which the source and receiver can also be placed anywhere in the stack of
layers. They use a 61\,pt Gauss-Kronrod quadrature for the Hankel transform.

% 1.5 Claim
With empymod I present a 1D forward modeling code that is based on
\cite{GEO.15.Hunziker} for the wavenumber-frequency domain calculation, and
\new{mainly} on \cite{GEO.12.Key} for the Hankel and Fourier transforms.
\old{As I will therefore refer to these publications quite a lot, I use the
name Hun15 to denote Hunziker et al. (2015), and the names Key09 and Key12 to
denote Key (2009, 2012). To denote the FHT filters as published by Key09 and
Key12 I will add the corresponding filter-size, e.g. Key09-401.}
In addition \new{to QWE and FHT}, empymod includes \new{an
adaptive quadrature (QUAD) for the Hankel transform, and the standard fast
Fourier transform FFT as well as} the logarithmic fast Fourier transform
FFTLog of \cite{RAS.00.Hamilton} \old{as a third} \new{for the} Fourier
transform \old{possibility to FHT and QWE for the frequency-to-time
transformation}.

To my knowledge, empymod is the first code that is freely available which
calculates the full wavefield for a layered-earth model with vertical isotropy
and has \old{both quadrature and filters} \new{different types of Hankel
transforms (QWE, QUAD, FHT) and various methods for the Fourier transform
(FFTLog, sine/cosine-filters, QWE, FFT)}\old{built in}. \old{This makes}
\new{An interesting use-case for} empymod \old{ideal not just for} \new{is
therefore} comparison studies of the \old{two} \new{different} methods,
\old{but also to build} \new{up to building} hybrid inversion schemes.
\old{Further advantages of empymod are:} \new{The code is published under the
lax permissive \emph{Apache Version 2.0 license}, which makes it available to
everyone\new{ for free}, even for commercial purposes. It therefore might
prove to be valuable for students and professionals alike. It is written in
Python, a modern, cross-platform, free and open-source programming language.
With its scientific libraries, mainly the numeric and scientific modules
\texttt{NumPy} and \texttt{SciPy}, it creates an extremely powerful numerical
calculation stack: Even though it is an interpreted language it can be very
fast, as \texttt{NumPy}/\texttt{SciPy} use under the hood routines in Fortran
or in C (using for instance the BLAS/Lapack libraries) for the linear algebra
computations, and Python is merely the glue. The comparisons show that
empymod is as fast as the existing codes. The code is vectorized (avoiding
loops) as much as possible, which improves computation. The most time-consuming
calculations have furthermore a flag to run parallelized. However, on a larger
scale such as an inversion routine one would probably want to parallelize the
kernel calls instead of the kernel itself. The code is hosted on GitHub, which
makes it easy for anyone to improve and contribute to the code:
\url{https://github.com/prisae/empymod}. The documentation is hosted on
\url{https://empymod.readthedocs.io}.}
\cite{TLE.17.Werthmuller}\new{ gives a tutorial-style introduction to EM modeling
using empymod.}


\old{
\textbf{(1) Python:} Python is a modern, cross-platform, free and open-source
programming language. With its scientific libraries, mainly the numeric and
scientific modules \texttt{NumPy} and \texttt{SciPy}, it creates an extremely
powerful numerical calculation stack.
}

\old{
\textbf{(2) Libre (free and open-source):} The code is published under the lax
permissive \emph{Apache Version 2.0 license}, which makes it available to
everyone, even for commercial purposes. It therefore might prove to be valuable
for students and professionals alike. Not only the code is free, but also the
underlying platform (Python), contrary to, for instance, Matlab. As such it
is ideal for reproducible research.
}

\old{
\textbf{(3) Lean:} The codebase is very lean. It is built from scratch, and it
therefore does not suffer the problems that sometimes come with the organic
growth of a codebase. The code follows as much as possible the \emph{DRY}
coding paradigm, \emph{Don't Repeat Yourself}.
}

\old{
\textbf{(4) Fast:} The code is vectorized \new{(avoiding loops)} as much as
possible, which improves computation. The most time-consuming calculations have
furthermore a flag to run parallelized. However, on a larger scale such as an
inversion routine one would probably want to parallelize the kernel calls
instead of the kernel itself. Even though Python is an interpreted language it
can be very fast, as \new{under the hood routines} \old{the underlying routines
are} in Fortran or in C (using for instance the BLAS/Lapack libraries)
\new{are used for the linear algebra computations}, and Python is merely the
glue. The comparisons show that empymod is as fast as the existing codes.
}

\old{
\textbf{(5) Community:} The code is hosted on GitHub, which makes it easy for
anyone to improve and contribute to the code, and will allow empymod to
grow:\\
\url{https://github.com/prisae/empymod}.
}

\old{
\textbf{(6) Well documented:} The code is extensively documented, and large
parts of it are extracted using automated tools (Sphinx) to create a manual:\\
\url{https://empymod.readthedocs.io}.
}

\old{These points make this a worthwhile extension to the existing codes from
Hun15 and Key12. The existing codes are written in Fortran, which requires
much more time than Python to get started for students or to develop for
anyone, or Matlab, which is a proprietary language. And the existing codes are
in static repositories where one can only download the code, which makes
interaction, contribution, or bug filing more difficult.}

After introducing the code in the first part I will present some comparisons by
reproducing results from the publications this code is based upon: First
\old{a} \new{some} comparison\new{s} to the analytical half-space solution,
followed by a comparison to EMmod by \old{Hun15}\cite{GEO.15.Hunziker}, and
finally a comparison to some results presented by \old{Key12}\cite{GEO.12.Key}.
\new{Last is a time-domain example in which I compare the four different
  Fourier transforms.}


\section{About the code}

The code consists of 5 files; these are 3 core modules plus \emph{utils}, which
contains input checks and other utilities, and \emph{filters}, containing the
FHT filter coefficients. The three core routines are: (1) \emph{kernel}, where
the wavenumber-domain calculation is carried out; (2) \emph{transform}, where
the Hankel and Fourier transforms are computed; and (3) \emph{model}, which
contains the actual modeling routines for end users.
\new{The main modeling routine is \emph{bipole}, which can calculate frequency-
and time-domain responses for arbitrary oriented, electric or magnetic bipole
sources and receivers of finite length.}
\old{As of now there are two main modeling routines implemented,
\emph{frequency} and \emph{time}, with which one can calculate the frequency-
and time-domain responses for electric or magnetic point sources and receivers,
directed along the three principal axis $x, y,$ and $z$. More modeling routines
can easily be added to \emph{model}, such as finite bipoles or arbitrary source
and receiver directions, as they do not affect the core of the calculation,
hence not \emph{kernel} nor \emph{transform}.}

The wavenumber-domain calculation in \emph{kernel} follows
\old{Hun15}\cite{GEO.15.Hunziker}, and calculates as such the complete
wavefield for a layered VTI model\new{.} \old{,
where the} \new{The} code makes no assumptions about the model\old{.}\new{:}
Source and receiver can be placed anywhere in the model\old{.}\new{, including
first and last layer, and you have to define if the first layer is air or
not. Bipoles can cross layer boundaries.} Depths, frequencies, and
source-receiver configuration have to be defined, and each layer is
characterized with its horizontal resistivity $\rho_\textrm{h}$, its electrical
anisotropy $\lambda$, where $\lambda = \sqrt{\rho_\textrm{v}/\rho_\textrm{h}}$,
its horizontal and vertical magnetic permeabilities $\mu_\mathrm{h}$ and
$\mu_\mathrm{v}$, and the horizontal and vertical electric permittivities
$\epsilon_\mathrm{h}$ and $\epsilon_\mathrm{v}$. The main difference\old{s}
between EMmod and empymod \new{is}\old{are the programming language and} the
Hankel transform, and a few additional fundamental differences: EMmod uses 2nd
order Bessel functions of the first kind $J_2$.  Published FHT filters
generally provide coefficients for 0th and 1st order Bessel functions $J_0,
J_1$ only.  Therefore, the recurrence relation
%
\begin{equation}
  J_2(kr) = \frac{2}{kr}J_1(kr) - J_0(kr)
  \label{eq:j2}
\end{equation}
%
is used, where $k$ and $r$ are the wavenumber and space-domain parameters,
respectively.
\new{Other differences are that empymod includes Fourier transforms, hence
time-domain calculation, and can model arbitrary rotated, finite bipoles.}
\old{Another difference is vectorization: EMmod carries out the
calculation in the wavenumber-domain by looping over frequencies, offsets, and
wavenumbers. This is carried out in a single calculation without looping in
empymod. Another difference is that empymod is much leaner than EMmod.
EMmod grew organically, as confirmed by the author, adding more and more
features, where empymod was designed with all 36 source-receiver
configurations from the beginning. As an example, the full-space solution in
empymod is one function (107 lines of code) for all source-receiver
configurations, whereas in EMmod it is split into no less than 16 functions
(364 lines of code). This should help to maintain the code easier, and it
should also be easier for new contributors to read into the code.}

The \new{QWE and filter} Hankel and Fourier transforms in \emph{transform}
follow \old{Key12}\cite{GEO.12.Key}, with a few changes. The most important
ones regarding speed is vectorization\new{, hence avoiding loops,} and a
splined version for the filter method, in addition to the traditional lagged
version: The lagged version samples the wavenumber from the minimum to the
maximum required wavenumber given the required offsets and the chosen filter
base with the spacing as defined by the filter. It subsequently carries out the
Hankel transform, and interpolates for the required offsets in the space
domain. The splined version, on the other hand, uses a user-specified number of
values per decade from the minimum to the maximum wavenumber. It then
interpolates for the required wavenumber values, and does the Hankel transform
afterwards. The lagged version is \emph{very} fast. However, with the splined
version a speed-up can be achieved in comparison with the original FHT at
higher precision if compared with the lagged version. All filters published
\new{in the source codes of }\old{by} \old{Key09 and Key12}\cite{GEO.09.Key,
GEO.12.Key} are included in empymod, which includes Key's own filters as well
as the filters by \cite{TMS.82.Anderson} and by \cite{GP.07.Kong}.  \new{(I
refer throughout to paper to a specific filter with author, year, and
filter-size, for instance Key09-201.)}

The most import part of \old{Key12}\new{GEO.12.Key} is the introduction of a
new quadrature algorithm to geophysics, named
qua\-dra\-ture-with-ex\-tra\-po\-la\-tion (QWE). QWE is a fast quadrature
method using the Shanks transformation \citep{JMP.55.Shanks} computed with
Wynn's epsilon algorithm \citep{MC.56.Wynn}. The advantage of quadrature over
filters is the ability to estimate the error. QWE continues until the absolute
error, estimated by the difference of subsequent iterations $n$, satisfies the
inequality
%
\begin{equation}
  |S^*_n-S^*_{n-1}| \le \varepsilon_r|S^*_n| + \varepsilon_a\ ,
  \label{eq:err}
\end{equation}
%
where $S^*$ is the extrapolated result, and $\varepsilon_r, \varepsilon_a$ are
the relative and absolute tolerance, respectively.

Whereas for the Hankel transform the \old{two}\new{three} methods QWE\new{,}
\old{and} FHT\new{, and QUAD, a standard adaptive quadrature using QAGSE from
the Fortran QUADPACK library,} are implemented, for the Fourier transform there
are \old{three}\new{four} methods implemented: QWE, FHT with the sine and
cosine filters, \new{the standard FFT,} and FFTLog \citep{RAS.00.Hamilton}.
The logarithmic fast Fourier transform FFTLog is ideal for this operation, as
generally a wide range of frequencies is required to go from \new{the}
frequency to \new{the} time domain. The FFTLog can yield faster results than
the QWE method and more precise than the FHT results, specifically for land
impulse responses at early times.

\old{In addition to} \new{Calculation time can be reduced by using} the splined
version of the QWE and the splined and lagged versions of the FHT\old{,
a}\new{. In addition, a}ll time-consuming calculations are set up to be able to
run in parallel, with the help of the Python-module numexpr.  This can
significantly speed up calculations if you run big models with many layers and
for many offsets and frequencies. However, if you include empymod in an
inversion scheme then it might be better to parallelize the calls to empymod,
instead of empymod itself.

It is important to note that calculations in the wavenumber domain depend
\emph{only} on offset $r = \sqrt{x^2+y^2}$; the scaling factor, which depends
on the angle $\varphi = \rm{atan2}(y, x)$, is multiplied afterwards. In order
to calculate for instance a circle around the source, the kernel has to be
called only once, and subsequently scaled by the angle-dependent factor for
each source-receiver pair. This makes the splined version very powerful, as it
can be used for irregularly distributed data\new{ as long as all receiver
depths are the same}.

\old{The code distributed on GitHub contains Jupyter Notebooks to reproduce the
results in this article, as well as some basic tests and benchmarks, and the
\LaTeX-source of the article itself.}

The run-time comparisons tests were run on a Lenovo ThinkCentre running Ubuntu
16.04 64-bit, with 8\,GB of memory and an Intel Core i7-4770 CPU @ 3.40GHz x
8\old{ (4 cores with hyper-threading)}. Comparing run times is always a
difficult task, and between different languages even more, here between
Python \new{(empymod)},
Fortran \new{(EMmod, Dipole1D)},
and Matlab \new{(scripts from \cite{GEO.12.Key})}.
However, they do serve as comparison.  I used the
Matlab-\texttt{timing} and Python-\texttt{timeit} functions, which behave
very similar. It is probably expected today, but still worth mentioning, that
the calculation is carried out in double precision.

\section{Comparison to analytical half-space solution}

\cite{PIER.10.Slob} published analytical frequency- and time-domain solutions
for the diffusive electric field in a VTI half-space\new{, where source and
receiver can be placed anywhere in the half-space}. As an example, I use the
same model as \old{Hun15}\cite{GEO.15.Hunziker} in their Figures~1 and~2: A
half-space with horizontal resistivity $\rho_\mathrm{h} = 1/3\,\Omega$\,m,
anisotropy $\lambda = \sqrt{10}$, and for frequency $f = 0.5\,$Hz. The source
is located horizontally at the origin \new{($x_\mathrm{s} = y_\mathrm{s} =
0\,$m)} at a depth of 150\,m, and the depth of the receivers is 200\,m. The
field is calculated on a regular grid with a spacing of 10\,m.
Figure~\ref{fig:analytical} shows the analytical amplitude and phase results
for the first quadrant (the other quadrants are simply symmetric copies of it).
%
\plot*[btp]{analytical}{width=\textwidth}{Analytical solution for a
half-space with $\rho_\mathrm{h} = 1/3\,\Omega$\,m, $\lambda = \sqrt{10}$, $f =
0.5\,$Hz, $z_\mathrm{s} = 150\,$m, and $z_\mathrm{r} = 200\,$m, calculated on a
regular grid with a spacing of 10 meters; (a) amplitude and (b) phase.}
%
The figure shows the result for x-directed electric source and receiver
dipoles ($G^\mathrm{ee}_{xx}$), other configurations yield similar results (the
choice is based on the insight that this is the most interesting of all
electric source to electric receiver fields, as can be seen in
\old{Hun15}\cite{GEO.15.Hunziker} Figures~1 and~2).

The error of the amplitude is shown in Figure~\ref{fig:onederror-amplitude} for
different settings regarding the Hankel transform: (a) for a 51\,pt QWE with
relative tolerance $\varepsilon_r$ and absolute tolerance $\varepsilon_a$ of
$10^{-12}$ and $10^{-30}$, respectively; (b) for a 21\,pt QWE with
$\varepsilon_r = 10^{-8}$ and $\varepsilon_a = 10^{-30}$; (c) for a 15\,pt QWE
with $\varepsilon_r = 10^{-8}$ and $\varepsilon_a = 10^{-18}$; (d) using the
standard FHT method with the filter Key09-\old{4}\new{2}01; (e) using the
splined FHT of the same filter with 40 points per decade; and (f) using the
lagged FHT of the same filter.
%
\plot*[btp]{onederror-amplitude}{width=.92\textwidth}{Error levels for different
  Hankel transform settings, compared with the analytical solution in Figure~1.
  The high precision QWE (a) and the standard FHT (d) have very low error
  levels, generally in the order of $10^{-6}$\,\% or less. The lagged FHT has
  much higher error levels, and the effects of interpolation can be seen
  clearly in the ring-like structure. However, almost the entire error is below
  0.1\,\% ($10^{-1}$\,\%), only the dark black parts have an error of 1\,\% or
  slightly more. \new{Hankel arguments for QWE: [rel.\ tol., abs. tol., nr of
  pts]; for splined FHT: [pts/decade].}}
%
The results from the high precision QWE (a) and the standard FHT (d) are almost
identical, even though they use completely different Hankel transforms.
\old{This can be seen better in Figure~\ref{fig:onederror-amp1d}, which shows a
cross-section through subplots (a), (c), (d), and (f) of
Figure~\ref{fig:onederror-amplitude} at a crossline offset of 4\,km.
%
% \plot[btp]{onederror-amp1d}{width=.5\textwidth}{\old{Relative percentage error
%   for crossline offset of 4\,km as shown in
%   Figure~\ref{fig:onederror-amplitude} (a), (c), (d), and (f).} Note: This
%   figure is removed.}
% %
}
The error is, in this case, not due to the method used for the Hankel transform,
but rather to the \old{limitations in computation: either because we reach the
numerical noise level, or because of }distinct differences between the
analytical solution, which uses the diffusive approximation, and the numerical
code, which calculates the complete field. If the QWE is calculated for lower
tolerance levels, as shown in (b) and (c), or the FHT is used with
interpolation as in (e) and (f), artefacts seem to arise from the Hankel
transform and the interpolation.  The error of the phase is shown in
Figure~\ref{fig:onederror-phase}, with the same conclusion.
%
\plot*[btp]{onederror-phase}{width=.92\textwidth}{Same as
  Figure~\ref{fig:onederror-amplitude}, but for the phase. Interesting is to
  see that interpolation in the wavenumber domain (e) yields slightly different
  patterns than interpolation in space domain (f).}
%

\new{Figure~\ref{fig:filtercomp-amplitude} shows a comparison of the error for
all nine included FHT-filters, for the amplitude (phase is not shown, but is
very similar). They are all very accurate, and by choosing an optimal filter
one has to decide between accuracy and speed, as shorter filters are faster.
However, we have to keep in mind that the accuracy of the filters depends on
the model, the offsets, and the frequencies. If we do not have an analytical
solution, as in this case, we cannot check the accuracy.
%
\plot*[btp]{filtercomp-amplitude}{width=.92\textwidth}{Comparison of the nine
  included FHT filters. All of them have an error mostly far below 1\,\%. For
  other models and other frequencies the result will be different. (Phase is
  not shown but looks very similar.)}
%
}

\section{Comparison to Hunziker et al. (2015)}

\old{Hun15}\cite{GEO.15.Hunziker} derive the electromagnetic fields in
astoundingly simple equations for all 36 possible source-receiver combinations
(electric and magnetic sources and receivers in three directions $x,y,z$) by
finding the solution for the vertical electric field and then applying the
duality principle and reciprocity to derive all components. The corresponding
code EMmod is published on the SEG website, and in \cite{GEO.16.Hunziker} they
published with iEMmod an inversion routine for it.

EMmod is written in Fortran and C. On execution, all parameters are
provided in an input file, and the resulting responses are written to an output
file.  The calculation is carried out for a regular grid in the space domain,
with at least two points in each direction. The code calculates \old{in loops}
the solution for all wavenumbers for the first quadrant, carries out the
wavenumber-to-space transformation, and then copies the result for the other
four quadrants, writing everything to the output file. This approach works very
well and even for millions of cells. However, the usage is probably more
academic. When it comes to actual measurements one deals with dozens to at most
hundreds of offsets, and usually on an irregular grid. Dipole1D and empymod
work more along the practical approach.

Figure~\ref{fig:emmod-HS} shows the error of amplitude and phase for EMmod.
On the left side with a colorscale from $10^0$ to $10^2$\,\% as used in
\old{Hun15}\cite{GEO.15.Hunziker}, on the right side with a colorscale
from $10^{-8}$ to $10^0$\,\% as used in Figures~\ref{fig:onederror-amplitude}
and~\ref{fig:onederror-phase}. Note that the error calculation in
\old{Hun15}\cite{GEO.15.Hunziker} was done slightly different. Here I
show the relative error between the analytical result and the result from
EMmod, where \old{Hun15}\cite{GEO.15.Hunziker} shows the relative error
between $\log_{10}$(analytical result) and $\log_{10}$(result from EMmod).
%
\plot*[btp]{emmod-HS}{width=.7\textwidth}{Error of EMmod for the half-space
  model in Figure~1. On the left side with the same colorscale as in
  \old{Hun15}\cite{GEO.15.Hunziker}, on the right side with the
colorscale as in Figures~2 and~3.}
%
Figure~\ref{fig:emmod-HS} shows a few interesting points. The responses from
EMmod in this example have significantly less precision than the results from
empymod. This does not mean in any way that EMmod is less precise, as EMmod
can be adjusted to yield much preciser results. However, this would
significantly increase the run time, so it has to be kept in mind for the run
time comparison. The important point is that EMmod does \emph{always} do an
interpolation in the space-frequency domain, by default, a linear
interpolation.  No interpolation is used in empymod \new{with QWE or FHT as
Hankel transform}, unless one specifies it to speed up the calculations
(splined QWE or splined and lagged FHT options), which will therefore yield
preciser results. This can be seen very nicely in the results.
Figures~\ref{fig:emmod-HS} (b) and (d) show white circles where the result is
very precise. These are the offsets close to those where the fields were
actually calculated. The black bands in-between are the areas where the result
was interpolated. It also shows that the spacing between calculated offsets
increases with increasing offsets. The error patterns from empymod in
Figures~\ref{fig:onederror-amplitude} and \ref{fig:onederror-phase} show
\new{in (a) and (d)} \old{generally} a different pattern, displaying \old{where
the code hits the numerical accuracy or} differences between the analytical,
diffusive solution and the numerical, full wavefield solution. It is only for
the splined and lagged versions that one sees the same, circular pattern\new{s}
appearing, which are due to interpolation.

Table~\ref{tbl:analytical} shows run times for these models. A few notes worth
mentioning: empymod carries out a lot of input checks, as it is quite
forgiving in what you input. EMmod, on the other hand, reads the input file
and writes the result to an output file, and copies the first quadrant to the
other three. Both have therefore some overhead, and the comparison is not
strictly 1:1. For the comparison the same model was used as shown above, on a
regular grid with a spacing of 100\,m.  On the test-machine the standard
\old{, vectorized,} QWE and FHT empymod would run into memory issues for denser
spacing, hence arrays of more than some 10\,000s of offsets.  The splined and
lagged versions of EMmod could handle it, however.
%
\tabl[btp]{analytical}{Run times for the half-space model shown in Figure~1, on
  a regular grid with spacing of 100\,m, for EMmod and different Hankel
  transform settings of empymod (times in milliseconds).}{
  \centering
  \begin{tabularx}{240pt}{rAAABBB}
  \toprule%
  EMmod & \multicolumn{3}{c}{QWE} & \multicolumn{3}{c}{FHT} \\
        & 1 & 2 & 3               & 1 & 2 & 3 \\
  \midrule%
   5040 & 10300 & 2530 & 1210 & 1480 & 875 & 6 \\
  \bottomrule%
  \end{tabularx}
}
%
QWE becomes faster for decreasing precision, not surprisingly, and the splined
and lagged version of FHT are faster than the standard version. The lagged FHT
is the fastest by quite a margin, and about 1/3 of that time is from the input
checks, so it takes only \old{about 4\,ms}\new{a few milliseconds} to calculate
the 11\,025 offsets.  What is interesting here is that the lagged FHT is still
more precise than EMmod with the given settings, which means a speed-up of a
factor 1000 for the same result.

Many 1D CSEM codes use the diffusive approximation that is valid for low
frequencies. The appealing part of the derivation of
\old{Hun15}\cite{GEO.15.Hunziker} is that it models the complete wavefield,
hence it is \new{also} valid for high frequencies and therefore wave-phenomena.
As an extreme case, \old{Hun15}\cite{GEO.15.Hunziker} show a 1D example for
ground-penetrating radar with a center frequency of 250\,MHz. To do so,
\new{they calculate} 2048
frequencies in the range of 1\,MHz to 2048\,MHz \old{are calculated} for the
Fast Fourier transform from frequency to time domain.  \new{Here I only
calculated the FFT with 850 frequencies from 1\,MHz to 850\,MHz and then
zero-padded up to 2048\,MHz, for EMmod and empymod; tests have shown that
this is sufficient.} Figure~\ref{fig:gpr} shows \new{the model with the survey
setup and} the \new{four}\old{three} results for EMmod, \new{QUAD for
relative and absolute tolerance of $10^{-12}$ and $10^{-20}$, respectively,}
empymod with a 21\,pt QWE and relative and absolute tolerance of $10^{-10}$
and $10^{-18}$, respectively, and empymod with FHT with the filter Key09-401,
together with the analytical solution for the arrival of the direct wave (red),
the wave refracted at the surface (\old{cyan}\new{yellow}), and the wave
reflected at the subsurface interface (\old{magenta}\new{green}).  \old{For the
results with empymod I used the regular FFT as Hun15, not FFTLog.}

\old{EMmod does clearly do the best job. QWE has troubles at very short
offsets (< 0.1\,m), and a \emph{noisy triangle} expanding from the origin.
This triangle could be narrowed by increasing the precision of the QWE, at the
cost of computation time. However, I was not able to completely get rid of it
with QWE.} \new{EMmod and the QUAD and QWE transforms of empymod yield pretty
much the same result. It is only at later times that some noise can be seen,
different in each method.} FHT shows the first arrivals well, however,
afterwards the result becomes extremely noisy.
%
\plot*[btp]{gpr}{width=1\textwidth}{GPR example for (a) EMmod, \new{(b) QUAD
  for relative and absolute tolerance of $10^{-12}$ and $10^{-20}$,
  respectively, (c) empymod with 51\,pt QWE for relative and absolute
  tolerance of $10^{-8}$ and $10^{-15}$, respectively, and (d)}\old{(b)
  empymod with 51\,pt QWE for relative and absolute tolerance of $10^{-10}$
  and $10^{-18}$, respectively, and (c)} empymod with FHT (Key09-401); run
  times are given in the subplot titles.}
%
Interesting are the calculation times for these three results. EMmod took
\old{more than 32}\new{roughly 9.5} hours to calculate,
\new{empymod with QUAD about 8.8 hours,}
empymod with QWE about
\old{34 minutes}\new{4.4 hours}, and empymod
with FHT \old{a bit over 4}\new{under a} minute\old{s}.
\new{Note that QWE has internally a check: If the kernel is too steep within
given intervals, it uses QUAD instead, as QWE cannot handle very steep
functions. QWE uses in about 1/3                                        % (37%)
of the calculation of this example QUAD.} Surprising is how good the FHT result
is, given that the filter was designed for CSEM data, hence frequencies 6 to 9
orders of magnitudes smaller \new{and
much bigger offsets}.  \old{One could implement the Gauss-Kronrod quadrature
into empymod, and see how this would compare to EMmod for GPR data in terms
of speed and precision. But, i}\new{I}n any case, EMmod/empymod are not
optimized for nor intended to model GPR data, and calculations in time-domain
will be much faster and more precise. It is nevertheless an interesting proof
of concept and shows that EMmod/empymod model the entire EM field.  \new{This
leaves the question if it would be possible to create a filter that works for
the frequencies and the offsets required for GPR calculations.}


\section{Comparison to Key (2009, 2012)}

\old{Key12}\new{GEO.12.Key} not only introduces QWE to geophysics, but also
compares QWE to FHT for various filters, some of which were specifically
created for the comparison. In order to calculate the models he translates
parts of Dipole1D (\old{Key09}\cite{GEO.09.Key}) from Fortran to Matlab. The
models are therefore isotropic and use the diffusive approximation. He
summarizes succinct and clear the FHT method\new{,} \old{and} outlines the QWE
method, and made the codes available from the SEG website.

The inclusion of the filter method FHT and the quadrature method QWE in
empymod follows \old{Key12}\cite{GEO.12.Key}, with one important exception:
vectorization. The Matlab code of \old{Key12}\cite{GEO.12.Key} loops over
frequencies, offsets, and wavenumbers; the code was developed to compare the
different hankel transforms, not with speed in mind.  In empymod both methods
are vectorized, hence various \old{frequencies and offsets} \new{offsets, and
in the case of FHT also various frequencies,} can be calculated for all
required wavenumbers in one calculation, which changes the conclusion drawn in
\old{Key12}\cite{GEO.12.Key} comparing the speed of QWE and FHT.
The vectorized approach is much faster but is limited by the memory of the
computer. If the model becomes too big, which depends on numbers of layers,
frequencies, offsets, and wavenumbers, the calculation has to be carried out by
looping over frequencies or offsets or both; the code never loops over
wavenumbers.

In \old{Key12}\cite{GEO.12.Key}, the standard QWE is always faster than the standard FHT. In the
vectorized version of empymod, the standard QWE \emph{can} be faster than the
standard FHT if the model is big and many offsets are required, but often it is
slower. Furthermore, the lagged FHT is generally much faster than the QWE
method, splined or not.

Table~\ref{tbl:runtimes} lists the run times for exactly the same models as
\old{Key12}\cite{GEO.12.Key} compared in his Table~1\old{.}\new{: 1\,km water layer, followed by a
1\,km overburden of 1\,$\Omega\,$m, 100\,m reservoir of 100\,$\Omega\,$m, and
lastly 1 or 96 layers of underburden with a resistivity of 1\,$\Omega\,$m;
frequency is 1\,Hz, source depth 990\,m, and receivers are on the sea-bottom;
there are 1, 5, 21, 81, or 321 offsets between 0.5 and 20\,km.} In this
comparison, a 9\,pt QWE is compared to a 201\,pt (Key12-201) and a 801\,pt
filter (\old{Key12}\new{Anderson82}-801), where the absolute and relative
tolerance are set so that the QWE achieves a similar error-level as the filters
($\varepsilon_r = 10^{-6}$ for the standard QWE and $10^{-2}$ for the splined
version, $\varepsilon_a = 10^{-24}$ in both cases). In order to make a fair
comparison I re-run the script from \old{Key12}\cite{GEO.12.Key}, and the run
times on the test-machine are roughly twice as fast as in the original paper.
Next to it are the results from empymod. It can be seen from the results that
empymod is significantly faster.%
%
\tabl[btp]{runtimes}%
{Run times comparing \new{the codes from \cite{GEO.12.Key}}\old{Key12} (Key)
and empymod (Wer) for the same cases as in (Key), Table~1 (in milliseconds).}{
  \centering
\begin{tabularx}{1.05\textwidth}{ccYYWWYYWWYYWW}
  \toprule
  %
  & & \multicolumn{4}{c}{QWE: 9\,pt}&\multicolumn{4}{c}{FHT: 201\,pt filter}&
  \multicolumn{4}{c}{FHT: 801\,pt filter} \\
  %
  \cmidrule(rl){3-6} \cmidrule(rl){7-10} \cmidrule(rl){11-14}
  %
  & &
  \multicolumn{2}{>{\columncolor{MyGray}\arraybackslash}c}{Normal} &
  \multicolumn{2}{c}{Splined} &
  \multicolumn{2}{>{\columncolor{MyGray}\arraybackslash}c}{Normal} &
  \multicolumn{2}{c}{Lagged} &
  \multicolumn{2}{>{\columncolor{MyGray}\arraybackslash}c}{Normal} &
  \multicolumn{2}{c}{Lagged} \\
  %
  Lay& Off& Wer& Key& Wer& Key& Wer&  Key& Wer& Key& Wer&  Key& Wer& Key\\
  \midrule
  %
    5&   1&   7&  16&   2&  28&   1&   19&   1&  20&   2&   76&   2&  76\\
    5&   5&  13&  69&   4&  29&   3&   92&   1&  24&   7&  365&   3&  82\\
    5&  21&  19& 300&   6&  36&   8&  384&   2&  25&  25& 1522&   3&  83\\
    5&  81&  37&1154&  16&  62&  30& 1475&   2&  28& 100& 5865&   3&  86\\
    5& 321& 108&4581&  58& 165& 124& 5867&   2&  41& 437&23165&   3&  99\\
  \midrule
  100&  21& 118& 463&  13&  51&  92&  604&   8&  37& 304& 2402&  19& 127\\
  100&  81& 296&1792&  23&  75& 337& 2313&   8&  40&1234& 9215&  19& 129\\
  100& 321&1021&7226&  65& 178&1408& 9350&  10&  53&5448&36520&  19& 142\\
  %
  \bottomrule
\end{tabularx}}%
%
Important to note is, however, not the absolute speed of empymod, but the
difference between the various Hankel transform methods. The standard QWE
method, for a similar error level as FHT, is faster than the standard FHT mode
only for many offsets. More importantly, the lagged FHT is generally much
faster than any QWE. QWE is very useful to check the error level of a result.
If many kernel evaluations have to be carried out, and speed is of importance,
the lagged FHT is the preferred method. As \old{Key12}\cite{GEO.12.Key} loops over each wavenumber,
the difference between the vectorized and non-vectorized version can be best
seen in the long 801\,pt filter.

These results regarding speed of calculation do not change the fact that the
advantage of the QWE method is to have an estimate of the error. However, the
filters designed for CSEM problems seem to be very good at solving them, as can
be seen in the low error levels in Figures~\ref{fig:onederror-amplitude}
and~\ref{fig:onederror-phase} or in \old{Key12}\cite{GEO.12.Key}.


Table~\ref{tbl:runtimes2} lists the run times for the same model as before,
but for empymod and Dipole1D using the FHT method with the filter Key09-201,
the default filter in \new{both} Dipole1D \new{and empymod}. For the regular
versions \old{(-)} empymod and Dipole1D perform very similar; empymod is a
little faster for the smaller tests, Dipole1D is slightly faster for the bigger
tests. It can be seen from the results that the parallel optimisation
\old{(par)} in empymod can result in a significant speed-up.  If the lagged
convolution optimisation is chosen, empymod is significantly faster than
Dipole1D. In empymod, additional offsets come at basically no cost in the
lagged version, whereas Dipole1D still uses more time to calculate more offsets.
This is most likely due to the writing of the output file, that becomes bigger
with bigger offsets.
%
\tabl[btp]{runtimes2}
{Run times comparing Dipole1D (\old{Key09}\cite{GEO.09.Key}) and empymod, using the filter Key09-201
  (default in Dipole1D; times in milliseconds). Optimisation: [-] None,
  [par] parallel, [spl] lagged FHT.}{ \centering
  \begin{tabularx}{.5\textwidth}{ccAAABB}
  \toprule
  %
     &    & \multicolumn{3}{c}{empymod} & \multicolumn{2}{c}{Dipole1D}\\
  %
  \cmidrule(rl){3-5} \cmidrule(rl){6-7}
  %
  Lay& Off&  - & par & spl &   - & spl\\
  \midrule
  %
    5&   1&    1&   2& 1&    4&  4\\
    5&   5&    3&   3& 2&    6&  5\\
    5&  21&    9&   5& 2&   12&  8\\
    5&  81&   30&  10& 2&   36& 18\\
    5& 321&  124&  34& 2&  130& 56\\
  \midrule
  100&  21&   91&  53& 9&   90& 13\\
  100&  81&  340& 106& 9&  333& 23\\
  100& 321& 1423& 313& 9& 1321& 62\\
  %
  \bottomrule
\end{tabularx}}%
%
Dipole1D was compiled using the free and open-source \emph{gfortran} compiler.
Compiling it with the (proprietary) \emph{ifort} compiler might result in
slightly faster run times. \new{Note that Dipole1D always calculates all six
receiver components, whereas empymod only calculates the required
component(s). On the other hand empymod computes the whole, anisotropic
wavefield, whereas Dipole1D is isotropic and uses the diffusive approximation.}

\new{
\section{Comparison of different Fourier transforms}
}

\new{
The following is a comparison of different Fourier transforms with the
analytical solution. The electromagnetic impulse response of an isotropic
half-space model below air with interface at $z=0\,$m for an x-directed
electromagnetic source at the origin
($x_\textrm{s}=y_\textrm{s}=z_\textrm{s}=0\,$m) and a receiver at an inline
offset of 6\,km at the surface ($r = x_\textrm{r}=6\,$km,
$y_\textrm{r}=z_\textrm{r}=0\,$m) is given by
\begin{equation}
  E_x(\rho,r,t) = \frac{1}{8} \sqrt{\frac{\mu_0^3}{\pi^3 t^5 \rho}}
                    \exp\left(-\frac{\mu_0 r^2}{4\rho t}\right)\ ,
  \label{eq:impulse}
\end{equation}
where $\rho$ is the halfspace resistivity, $\mu_0$ is the permeability of free
space, and $t$ is time (neglecting the impulse of the airwave at
$t=0\,$s), as given by \citet[][ eq. 5.38]{PhD.97.Wilson}.
}

\new{
Figure~\ref{fig:impulse} shows the result of this comparison, where FFTLog
with 15\,ms is the fastest method and FFT with 666\,ms the slowest one.
Interesting is to see how many frequencies are used internally, as listed in
Table~\ref{tbl:time}.
%
\tabl[btp]{time}{Run-times for the four different Fourier transforms, and
frequency range they require. More frequencies does not necessarily mean faster
nor more precise. All Fourier transform do interpolation either in frequency or
in time domain.}{
  \centering
  \begin{tabular}{lcccc}
  \toprule
  %
   Method  &  \# freq   & min freq & max freq & time \\
           &            & Hz       & Hz       & ms \\
  %
  \midrule
  %
    FFTLog     &  70& 1.8e-5& 1.4e2 &   6\\
    Sine-filter & 128& 5.3e-7& 5.7e3 &  11\\
    QWE        & 178& 7.9e-5& 6.3e4 & 298\\
    FFT        &  61& 5.0e-5& 5.2e1 & 562\\
  %
  \bottomrule
\end{tabular}}%
%
Even though FFT uses the least frequencies it is the slowest (and least
precise) method. The reason is that it only calculates the response for 61
frequencies, but for the actual FFT it afterwards interpolates them at $2^{20}
= 1\,048\,579$ frequencies to have a regular array from 5e-5\,Hz to 52\,Hz.
To get a better result we would have to get to lower and higher frequencies,
but then the number of required frequencies would even grow bigger.  Note that
this example is a difficult one, with source and receiver at the interface
between air and the subsurface. Models where source and receiver are not at the
same depth and not at the air-interface, for instance in marine CSEM where
source and receiver are in the water layer, are generally much easier, in other
words, faster and more precise.
}
%
\plot*[btp]{impulse}{width=\textwidth}{Comparison of different Fourier transforms for an impulse response at the interface between air and a halfspace. FFTLog
is the fastest, and FFT the slowest in this example.}
%

\section{Conclusions}

The presented code empymod is \new{an}\old{a fast, lean,} open-source
electromagnetic modeler \old{, well documented and hosted in a way that makes
collaboration easy.  It}\new{that} can model the full wavefield of a 3D source
in a VTI layered-earth. \new{Three different Hankel transform methods
(quadrature, quadrature-with-extrapolation, and filters) are
included into empymod, as well as four different Fourier transform methods
(FFT, FFTLog, quadrature-with-extrapolation, and sine/cosine-filters).}
Additional to the obvious application of modeling and inversion of
electromagnetic data, empymod can \new{therefore} be used to investigate into
the differences between \new{different Hankel and Fourier transforms, or
between different filters} \old{filters and quadrature} for full-wavefield
electromagnetic calculations over a wide range of frequencies \new{as well as
times}.  \old{There are not many full-wavefield codes openly available and, as
such, empymod with QWE and FHT is an important addition to EMmod, which uses
a 61\,pt Gauss-Kronrod quadrature.} \new{The examples show that empymod can
compete with existing code in both speed and accuracy, even though it is
written in a dynamically typed language.}

Outside of the traditional scope \old{I think} empymod can be very educative
for someone who is just starting to get interested in electromagnetic modeling
or even just in Hankel \new{and Fourier} transforms, hence for educational
purpose: it is \old{brilliant}\new{well suited} for students, specifically as
Python is becoming more and more widespread in academia, and the number of
Universities that teach geophysicist Python (mostly instead of Matlab) is
growing annually. Python, like Matlab, has the advantage of very fast
developing times, and considerably lower entry barriers than C or Fortran
for beginners.

There are many possibilities to improve empymod\old{,} or to add additional
functionalities\old{, as in any software (this is why it is important to keep
open-source code in version-controlled repositories such as GitHub instead in
static repositories). Possible additions are}\new{, for instance} more modeling
routines\old{,} such as \new{loop sources}\old{arbitrary source
and receiver dipole lengths, arbitrary source and receiver rotations, or
variable receiver depths within one calculation. Other features to include
could be further Hankel and Fourier transforms, such as the Gauss-Kronrod as in
EMmod}. \new{Improving code abstraction or creating a graphical user interface
are further possibilities. The code comes with a complete testing-suite, but
the included} \old{The} benchmarks \old{and tests included in the code} could
also be improved and extended \new{with regression tests}. \old{By the time
this article is published some of them might already be implemented by me or by
the community.}


\section{Acknowledgment}

I would like to thank the \emph{Consejo Nacional de Ciencia y Tecnología},
México (CONACYT) for funding this postdoc, the \emph{Instituto Mexicano del
Petróleo} (IMP) for allowing me to publish the code under a free software
license, and the entire electromagnetic research group of Aleksandr Mousatov at
the IMP for fruitful discussions.  I owe a special thanks to Jürg Hunziker for
answering all my questions regarding his code and publication\new{. I thank the
assistant editor, C. Torres-Verdin, the associate editor, F. Broggini, and Jan
Thorbecke, Vladimir Puzyrev, and a third, anonymous reviewer, as well as
Jürg Hunziker and Kerry Key, whose feedback greatly improved the clarity of the
manuscript.}\old{, and to both Jürg Hunziker and Kerry Key for feedback
regarding this manuscript that greatly improved the article.}


% REFERENCES
\bibliographystyle{seg}
\begin{thebibliography}{}
\itemsep0pt

\bibitem[Anderson, 1975]{TRP.75.Anderson}
Anderson, W.~L.,  1975, Improved digital filters for evaluating {F}ourier and
  {H}ankel transform integrals: Technical report, U.S. Geological Survey.
\newblock
  (\href{https://pubs.er.usgs.gov/publication/70045426}{https://pubs.er.usgs.gov/publication/70045426}).

\bibitem[Anderson, 1979]{GEO.79.Anderson}
--------, 1979, Numerical integration of related {H}ankel transforms of orders
  0 and 1 by adaptive digital filtering: Geophysics, {\bf 44}, 1287--1305.
\newblock (\href{http://dx.doi.org/10.1190/1.1441007}{doi: 10.1190/1.1441007}).

\bibitem[Anderson, 1982]{TMS.82.Anderson}
--------, 1982, Fast {H}ankel transforms using related and lagged convolutions:
  ACM Trans. Math. Softw., {\bf 8}, 344--368.
\newblock (\href{http://doi.acm.org/10.1145/356012.356014}{doi:
  10.1145/356012.356014}).

\bibitem[Anderson, 1984]{GEO.84.Anderson}
--------, 1984, {On: “Numerical integration of related Hankel transforms by
  quadrature and continued fraction expansion” by \cite{GEO.83.Chave}}:
  Geophysics, {\bf 49}, 1811--1812.
\newblock (\href{http://dx.doi.org/10.1190/1.1441595}{doi: 10.1190/1.1441595}).

\bibitem[Anderson, 1989]{GEO.89.Anderson}
--------, 1989, A hybrid fast {H}ankel transform algorithm for electromagnetic
  modeling: Geophysics, {\bf 54}, 263--266.
\newblock (\href{http://dx.doi.org/10.1190/1.1442650}{doi: 10.1190/1.1442650}).

\bibitem[Chave, 1983]{GEO.83.Chave}
Chave, A.~D.,  1983, Numerical integration of related {H}ankel transforms by
  quadrature and continued fraction expansion: Geophysics, {\bf 48},
  1671--1686.
\newblock (\href{http://dx.doi.org/10.1190/1.1441448}{doi: 10.1190/1.1441448}).

\bibitem[Chave, 2009]{GJI.09.Chave}
--------, 2009, On the electromagnetic fields produced by marine frequency
  domain controlled sources: Geophysical Journal International, {\bf 179},
  1429--1457.
\newblock (\href{http://dx.doi.org/10.1111/j.1365-246X.2009.04367.x}{doi:
  10.1111/j.1365-246X.2009.04367.x}).

\bibitem[Chave et~al., 1991]{B.SEG.91.Chave}
Chave, A.~D., S.~C. Constable, and R.~N. Edwards,  1991, Electrical exploration
  methods for the seafloor, {\it in} Electromagnetic Methods In Applied
  Geophysics Vol.\ 2: SEG, Investigations in Geophysics, No.~3, 12,  931--966.
\newblock (\href{http://dx.doi.org/10.1190/1.9781560802686}{doi:
  10.1190/1.9781560802686}).

\bibitem[Edwards, 2005]{SG.05.Edwards}
Edwards, R.~N.,  2005, Marine controlled source electromagnetics: principles,
  methodologies, future commercial applications: Surveys in Geophysics, {\bf
  26}, 675--700.
\newblock (\href{http://dx.doi.org/10.1007/s10712-005-1830-3}{doi:
  10.1007/s10712-005-1830-3}).

\bibitem[Ghosh, 1971]{GP.71.Gosh}
Ghosh, D.~P.,  1971, The application of linear filter theory to the direct
  interpretation of geoelectrical resistivity sounding measurements:
  Geophysical Prospecting, {\bf 19}, 192--217.
\newblock (\href{http://dx.doi.org/10.1111/j.1365-2478.1971.tb00593.x}{doi:
  10.1111/j.1365-2478.1971.tb00593.x}).

\bibitem[Hamilton, 2000]{RAS.00.Hamilton}
Hamilton, A. J.~S.,  2000, Uncorrelated modes of the non-linear power spectrum:
  Monthly Notices of the Royal Astronomical Society, {\bf 312}, 257--284.
\newblock (\href{http://dx.doi.org/10.1046/j.1365-8711.2000.03071.x}{doi:
  10.1046/j.1365-8711.2000.03071.x}).

\bibitem[Hunziker et~al., 2016]{GEO.16.Hunziker}
Hunziker, J., J. Thorbecke, J. Brackenhoff, and E. Slob,  2016, Inversion of
  controlled-source electromagnetic reflection responses: Geophysics, {\bf 81},
  F49--F57.
\newblock (\href{http://dx.doi.org/10.1190/geo2015-0320.1}{doi:
  10.1190/geo2015-0320.1}).

\bibitem[Hunziker et~al., 2015]{GEO.15.Hunziker}
Hunziker, J., J. Thorbecke, and E. Slob,  2015, The electromagnetic response in
  a layered vertical transverse isotropic medium: {A} new look at an old
  problem: Geophysics, {\bf 80}, F1--F18.
\newblock (\href{http://dx.doi.org/10.1190/geo2013-0411.1}{doi:
  10.1190/geo2013-0411.1}).

\bibitem[Key, 2009]{GEO.09.Key}
Key, K.,  2009, {1D} inversion of multicomponent, multifrequency marine {CSEM}
  data: {M}ethodology and synthetic studies for resolving thin resistive
  layers: Geophysics, {\bf 74}, F9--F20.
\newblock (\href{http://dx.doi.org/10.1190/1.3058434}{doi: 10.1190/1.3058434}).

\bibitem[Key, 2012]{GEO.12.Key}
--------, 2012, Is the fast {H}ankel transform faster than quadrature?:
  Geophysics, {\bf 77}, F21--F30.
\newblock (\href{http://dx.doi.org/10.1190/GEO2011-0237.1}{doi:
  10.1190/GEO2011-0237.1}).

\bibitem[Kong, 2007]{GP.07.Kong}
Kong, F.~N.,  2007, Hankel transform filters for dipole antenna radiation in a
  conductive medium: Geophysical Prospecting, {\bf 55}, 83--89.
\newblock (\href{http://dx.doi.org/10.1111/j.1365-2478.2006.00585.x}{doi:
  10.1111/j.1365-2478.2006.00585.x}).

\bibitem[Løseth and Ursin, 2007]{GJI.07.Loseth}
Løseth, L.~O., and B. Ursin,  2007, Electromagnetic fields in planarly layered
  anisotropic media: Geophysical Journal International, {\bf 170}, 44--80.
\newblock (\href{http://dx.doi.org/10.1111/j.1365-246X.2007.03390.x}{doi:
  10.1111/j.1365-246X.2007.03390.x}).

\bibitem[Nekut and Spies, 1989]{PIEEE.89.Nekut}
Nekut, A.~G., and B.~R. Spies,  1989, Petroleum exploration using
  controlled-source electromagnetic methods: Proceedings of the IEEE, {\bf 77},
  338--362.
\newblock
  (\href{http://ieeexplore.ieee.org/lpdocs/epic03/wrapper.htm?arnumber=18630}{doi:
  10.1109/5.18630}).

\bibitem[Shanks, 1955]{JMP.55.Shanks}
Shanks, D.,  1955, Non-linear transformations of divergent and slowly
  convergent sequences: Journal of Mathematics and Physics, {\bf 34}, 1--42.
\newblock (\href{http://dx.doi.org/10.1002/sapm19553411}{doi:
  10.1002/sapm19553411}).

\bibitem[Slob et~al., 2010]{PIER.10.Slob}
Slob, E., J. Hunziker, and W.~A. Mulder,  2010, Green's tensors for the
  diffusive electric field in a {VTI} half-space: PIER, {\bf 107}, 1--20.
\newblock (\href{http://dx.doi.org/10.2528/PIER10052807}{doi:
  10.2528/PIER10052807}).

\bibitem[Wait, 1982]{B.AP.82.Wait}
Wait, J.~R.,  1982, Geo-{E}lectromagnetism: Academic Press Inc.
\newblock ({I}SBN: 978-0127308807).

\bibitem[Wilson, 1997]{PhD.97.Wilson}
Wilson, A.~J.~S., 1997, The equivalent wavefield concept in multichannel
transient electromagnetic surveying: Ph.D., University Of Edinburgh.
\newblock (\href{http://hdl.handle.net/1842/7101}{uri:
hdl.handle.net/1842/7101}).

\bibitem[Wynn, 1956]{MC.56.Wynn}
Wynn, P.,  1956, {On a device for computing the $e_m(S_n)$ tranformation}:
  Math. Comput., {\bf 10}, 91--96.
\newblock (\href{http://dx.doi.org/10.1090/S0025-5718-1956-0084056-6}{doi:
  10.1090/S0025-5718-1956-0084056-6}).

\bibitem[Werthmüller, 2017]{TLE.17.Werthmuller}
Werthmüller, D., 2017, Getting started with controlled-source electromagnetic
1D modeling: The Leading Edge, {\bf 36}, 352--355.
\newblock (\href{http://dx.doi.org/10.1190/tle36040352.1}{doi:
  10.1190/tle36040352.1}).

\bibitem[Ziolkowski and Wright, 2012]{IEEE.12.Ziolkowksi}
Ziolkowski, A., and D. Wright,  2012, The potential of the controlled source
  electromagnetic method: A powerful tool for hydrocarbon exploration,
  appraisal, and reservoir characterization: Signal Processing Magazine, IEEE,
  {\bf 29}, 36--52.
\newblock (\href{http://dx.doi.org/10.1109/MSP.2012.2192529}{doi:
  10.1109/MSP.2012.2192529}).

\end{thebibliography}

\end{document}
